%%
%% derived from typeinst.tex from the LLNCS website
%%

\documentclass[natbib]{sigplanconf}

% The following \documentclass options may be useful:
%
% 10pt          To set in 10-point type instead of 9-point.
% 11pt          To set in 11-point type instead of 9-point.
% authoryear    To obtain author/year citation style instead of numeric.

\usepackage{amsmath,amssymb,stmaryrd}
\usepackage{proof}
\usepackage{url}
\usepackage{graphicx}

% for typesetting code
\usepackage{listings}
\lstset{ basicstyle=\ttfamily\small,showstringspaces=false,captionpos=b }

% comments
\newif\ifshowcomments
\showcommentstrue

\ifshowcomments
\newcommand{\mynote}[2]{\fbox{\bfseries\sffamily\scriptsize#1}
  {\small$\blacktriangleright$\textsf{\emph{#2}}$\blacktriangleleft$}}
\else
\newcommand{\mynote}[2]{}
\fi
\newcommand{\noteP}[1]{\mynote{Paul}{#1}}
\newcommand{\noteE}[1]{\mynote{Edwin}{#1}}

\begin{document}

% For the user-defined commands

%%%
%%% general things
%%%

%% spacing commands
\newcommand{\figfill}[1][14pt]{\hspace{#1}\hfill}
\newlength{\indlen}
\setlength{\indlen}{8pt}
\newcommand{\ind}[1]{\hspace{#1\indlen}}


%%%
%%% theorem environments
%%%

\newtheorem{lemma}{Lemma}
\newtheorem{theorem}{Theorem}
\newtheorem{definition}{Definition}


%%%
%%% proof environment
%%%
\newenvironment{proof}
               {\hspace{-1\parindent}\textbf{Proof:}}
               {\hfill $\square$}


%%%
%%% for writing grammars
%%%

% bor stands for bar-or ("or" written with vertical bars), for grammars
% \borstar has no spacing before it, for use at the beginning of a line
\newcommand{\borspacing}{\hspace{.5em}}
%\newcommand{\borstar}{\protect\rule[-2pt]{1pt}{1em}\hspace{1pt}\protect\rule[-2pt]{1pt}{1em}\borspacing}
\newcommand{\borstar}{\protect\rule[-2pt]{1pt}{1em}\borspacing}
\newcommand{\bor}{\borspacing\borstar}


%%%
%%% for repetitive commands that just set the font face
%%%

\newcommand{\definermcommand}[2]{\newcommand{#1}{\ensuremath{\mathrm{#2}}}}
\newcommand{\definesfcommand}[2]{\newcommand{#1}{\ensuremath{\mathsf{#2}}}}
\newcommand{\definebfcommand}[2]{\newcommand{#1}{\ensuremath{\mathbf{#2}}}}


%%%
%%% for lambda-calculus encoding
%%%

\newcommand{\lcenc}[2]{\llbracket #2 \rrbracket^{#1}_{\lambda}}
\newcommand{\lcenct}[1]{\llbracket #1 \rrbracket_{\lambda}}
\definesfcommand{\basetype}{b}
\definesfcommand{\basenu}{b_{\nu}}
\definesfcommand{\Dom}{Dom}
\definesfcommand{\Ran}{Ran}


%%%
%%% Stuff for formalizing the operational semantics
%%%

% "essential" operators
\newcommand{\op}{\ensuremath{op}}
\definesfcommand{\cmp}{cmp}
\definesfcommand{\nuOp}{nu}
\definesfcommand{\mbTop}{top}
\definesfcommand{\boundP}{boundP}
\newcommand{\case}[1][\vec{c}]{\ensuremath{\mathsf{case}_{#1}}}

% multi-binding operators
\definesfcommand{\emp}{emp}
\definesfcommand{\elimEmp}{elimEmp}
\definesfcommand{\comb}{comb}
\definesfcommand{\sep}{sep}

% necessary constructors for multi-bindings...
\definesfcommand{\Just}{Just}
\definesfcommand{\Nothing}{Nothing}
\definesfcommand{\Refl}{Refl}
\definesfcommand{\ICBase}{ICBase}
\definesfcommand{\ICCons}{ICCons}

% necessary constructors for single bindings
\definesfcommand{\True}{True}
\definesfcommand{\False}{False}

% term constructs
\newcommand{\ctorApp}[2]{#1(#2)}
\newcommand{\opApp}[2]{#1(#2)}
\newcommand{\lamabs}[3]{\lambda #1:#2.\, #3}
\newcommand{\lamabsnot}[2]{\lambda #1.\,#2}
\newcommand{\nuabs}[2]{\nu #1.\,#2}
\definesfcommand{\letname}{let}
\newcommand{\lete}[3]{\letname\;#1 = #2\;\mathsf{in}\;#3}
\newcommand{\leteArr}[3]{\begin{array}{@{}l@{}}\letname\;#1 = #2\;\mathsf{in}\\#3\end{array}}

% term contexts
\newcommand{\tctxt}[2]{\ensuremath{#1[#2]}}


% rewriting and equality
\newcommand{\rrto}{\longrightarrow}
\newcommand{\equiva}{\equiv_{\alpha}}

% alpha-well-formedness
\newcommand{\WFa}[1]{\ensuremath{\mathrm{WF}_\alpha(#1)}}

% type constructs; should we omit these?
% \definesfcommand{\Name}{Name}
% \newcommand{\To}{\Rightarrow}
% \newcommand{\tlamabs}[2]{\Lambda #1.\,#2}
% \newcommand{\tapp}[2]{#1[#2]}
% \newcommand{\lamgamma}[2][\Gamma]{\lambda #1\,.\,#2}
% \newcommand{\nuabsnot}[2]{\nu\,#1\,.\,#2}
% \newcommand{\ctype}[2]{\bigcirc_{#1} #2}


%%% Local Variables: 
%%% mode: latex
%%% TeX-master: "paper"
%%% End: 


%% names of things and a few other basic commands
\newcommand{\ourlib}{\text{\textsc{HobbitLib}}}
\newcommand{\fastourlib}{\text{\textsc{Fast HobbitLib}}}
\newcommand{\Fcc}{\ensuremath{F^{\bigcirc\mathrm{c}}}}
\newcommand{\Fnu}{\ensuremath{F^{\nu}}}

%% to declare whether we are building the tech report or not
\newif{\iftechreport}
\techreportfalse


\conferenceinfo{Haskell '11}{September 2011, Tokyo, Japan}
\copyrightyear{2011} 
\copyrightdata{978-1-4503-0860-1/11/09} 

%\titlebanner{banner above paper title}        % These are ignored unless
\preprintfooter{Hobbits for Functional Programming}   % 'preprint' option specified.

\title{Hobbits for Haskell: A Library for Higher-Order Encodings in Functional Programming Languages}

\authorinfo{Edwin Westbrook}
           {Rice University}
           {emw4@rice.edu}
\authorinfo{Nicolas Frisby}
           {University of Kansas}
           {nfrisby@ittc.ku.edu}
\authorinfo{Paul Brauner}
           {Rice University\\(Current affiliation: Google Inc)}
           {polux@google.com}

\maketitle

\begin{abstract}
  Adequate encodings are a powerful programming tool, which eliminate
  whole classes of program bugs: they ensure that a program cannot
  generate ill-formed data, because such data is not part of the
  representation; and they also ensure that a program is well-defined,
  meaning that it cannot have different behaviors on different
  representations of the same piece of data. Unfortunately, it has
  proven difficult to define adequate encodings of programming
  languages themselves. Such encodings would be very useful in
  language processing tools such as interpreters, compilers,
  model-checking tools, etc., as these systems are often difficult to
  get correct. The key problem in representing programming languages
  is in encoding binding constructs; previous approaches have serious
  limitations in either the operations they allow or the correctness
  guarantees they make. In this paper, we introduce a new library for
  Haskell that allows the user to define and use higher-order
  encodings, a powerful technique for representing bindings.  Our
  library allows straightforward recursion on bindings using
  pattern-matching, which is not possible in previous approaches.  We
  then demonstrate our library on a medium-sized example,
  lambda-lifting, showing how our library can be used to make strong
  correctness guarantees at compile time.
\end{abstract}

\category{D.3.3}{Programming Langauges}{Language Constructs and Features}

\terms
Languages, Design, Theory

%%\keywords
%%Haskell, Variable-Binding


%%%%%%%%%%%%%%%%%%%%%%%%%%%%%%%%%%%%%%%%
%%%%%%%%%%     Section        %%%%%%%%%%
%%%%%%%%%%%%%%%%%%%%%%%%%%%%%%%%%%%%%%%%
\section{Introduction}

Having the right data representation can vastly affect the ease with
which we write correct programs. Ideally, one would like to use a
representation that is an \emph{adequate encoding}, meaning that the
relationship between the data being represented and the
representations of that data is a one-to-one correspondence. Adequate
encodings eliminate whole classes of program bugs: they ensure that a
program cannot generate ill-formed data, because such data is not part
of the representation; and they also ensure that a program is
well-defined, meaning that it cannot have different behaviors on
different representations of the same piece of data.


Unfortunately, it has proven difficult to define adequate encodings
of programming languages themselves. Such encodings would be very
useful in language processing tools such as interpreters,
compilers, model-checking tools, etc., as these systems are often
difficult to get correct. The key problem in representing programming
languages is in encoding binding constructs, such as the
simply-typed $\lambda$-calculus function
\[
\lamabs{x}{A}{x}
\]
Such constructs bind a name, in this case $x$, that can only be used
inside the scope of the binding. Defining adequate encodings of
bindings is difficult because of $\alpha$-Equivalence and scoping,
which state respectively that bindings are equal up to renaming of
bound names and that names cannot occur outside of bindings.  For
example, straightforward representations based on de Bruijn indices do
not satisfy scoping because nothing in such representations prevents
an occurrence of an unbound name.

There has been much research into encodings for bindings in type
theory \citep{licata09,poswolsky08,pientka08,urban07}, logic
\citep{miller05,gabbaypitts02}, and programming languages
\cite{pitts10,mcbride04}.  So far, however, the only approaches that
can ensure both $\alpha$-Equivalence and Scoping --- and that can be
used in existing programming languages --- are typed de Bruijn indices
\citep{sheard05,atkey09} and Higher-Order Abstract Syntax (HOAS)
\citep{pe88,meijer-hutton95,fegaras96,washburn03}.  Both of these are
difficult to use: typed de Bruijn indices require a good deal of
arithmetic (both at the term and the type level); while HOAS requires
all operations on bindings to be expressed as folds, which cannot
directly express many desired operations (without adding significant
support to the programming language as done by e.g.\
\citet{pientka08}). In fact, the state of the art generally requires
using both representations, and switching between the two when
necessary \citep{carette09,atkey09}.


In this work, we describe a library, \ourlib\ (for Higher-Order
Bindings), that allows higher-order encodings in the Haskell
programming language. By a \emph{higher-order encoding}, we mean an
encoding of data with name-bindings in which each bound name has an
associated type in the host language (which is Haskell, in this case).
Readers familiar with HOAS will note that we use the term
``higher-order encoding'' in a more general sense than HOAS, where we
view the latter as a specific instance of the former. Higher-order
encodings allow us to build adequate encodings of typed programming
languages. For example, we can build an encoding using our approach
that does not allow the ill-typed (under simple typing) $\lambda$-term
$\lamabs{x}{A}{x\;x}$. This in turn ensures that any program we write
over such encodings, such as a compiler or interpreter, is
\emph{guaranteed} to produce well-formed, well-typed programs,
removing whole classes of bugs from such systems.


%  representation for bindings that is safe,
% efficient, and higher-order. This representation has been implemented
% in a library, called \ourlib\ (for Higher-Order Bindings), that can be
% downloaded from the authors' webpages.  By safe, we mean that it is an
% adequate encoding; specifically, names cannot escape their scopes. By
% efficient, we mean that our approach is faster than other (unsafe)
% methods that do work in the langauges of today. By higher-order, we
% mean that our approach allows types to be associated with
% names. Having a representation that is higher-order allows us to build
% adequate encodings of typed languages. For example, we can build an
% encoding using our approach that does not allow the ill-typed (under
% simple typing) $\lambda$-term $\lamabs{x}{A}{x\;x}$. Readers familiar
% with HOAS will note that we use the term ``higher-order encoding'' in
% a more general sense than HOAS, where we view the latter as a specific
% instance of the former.



% FIXME: Nominal Logic requires a permutation operation, except
% that we can't permute arbitrary types, so we instead make a
% representation with primitive ops that avoids permutation
At a high level, our approach is based on the Calculus of Nominal
Inductive Constructions (CNIC), a type theory of bindings
\cite{westbrook+09,westbrook-thesis}. CNIC encodes bindings with a
construct called the $\nu$-abstraction, a higher-order construct which
captures the expected properties of bindings and also allows useful
features such as pattern-matching on bindings and comparing names. At
a low level, we show here that $\nu$-abstractions can be defined in
functional languages like Haskell as pairs \lstinline{(n, body)} of a
fresh natural number \lstinline{n} and a body \lstinline{body} that is
allowed to use \lstinline{n} as a name.  This is based on the
observation by \citet{gabbaypitts02} that bindings are equivalent (in
Nominal Logic) to pairs of a fresh name and a body of the
binding. This definition makes it easy to define pattern-matching on
bindings, a necessary ingredient in defining operations on data with
bindings. In order to ensure that the number \lstinline{n} stays
fresh, we make these pairs opaque, allowing them to be manipulated
only via a suite of operations guaranteed to maintain freshness.

The remainder of the paper is organized as follows.  Section
\ref{sec:data-with-bindings} defines exactly what properties we expect bindings
to have and shows how our approach captures these properties, using the
simply-typed $\lambda$-calculus as an example.  Section
\ref{sec:matching-on-bindings} shows how to write operations, such as
capture-avoiding substitution, on data representations using our approach to
bindings.  Section \ref{sec:lambda-lifting} gives a more extended example,
lambda-lifting, implemented in \ourlib.  Section \ref{sec:implementation}
explores the definition of \ourlib\ in Haskell, which includes both unsafe type
casts and impure I/O operations. Section \ref{sec:safety} proves that our
approach is both type safe and pure, meaning both that the unsafe casts
used in \ourlib\ will never cause run-time failures and also that the impure
I/O operations cannot be observed at run-time.
% Section \ref{sec:performance} compares the performance of our approach
% to other methods.
Section \ref{sec:related-work} then compares to related work and
Section \ref{sec:conclusion} concludes.  All of the code has been
implemented and tested using GHC 7, and it can be downloaded from the
first author's webpage, at \lstinline{http://www.cs.rice.edu/~emw4/}.



%%%%%%%%%%%%%%%%%%%%%%%%%%%%%%%%%%%%%%%%
%%%%%%%%%%     Section        %%%%%%%%%%
%%%%%%%%%%%%%%%%%%%%%%%%%%%%%%%%%%%%%%%%
\section{Defining Data with Bindings}
\label{sec:data-with-bindings}

In this section, we introduce the notion of binding that \ourlib\
relies upon, and show how it can be used to define encodings of data
with bindings. In this paper, a \emph{binding} is a construct, such
as the $\lambda$-abstraction $\lamabs{x}{A}{t}$ of the simply-typed
$\lambda$-calculus, that introduces a fresh name in a local scope,
with the following four properties:
\begin{itemize}
\item \textbf{Freshness}: The name that is introduced is distinct from
  any names bound outside the given binding, so for example the terms
  $\lamabs{x}{A}{\lamabs{y}{B}{x}}$ and
  $\lamabs{x}{A}{\lamabs{y}{B}{y}}$ are distinct;
%
\item \textbf{$\alpha$-Equivalence}: Terms with name-bindings are
  equal up to renaming of bound names, so e.g.\ $\lamabs{x}{A}{x}$
  equals $\lamabs{y}{A}{y}$;
%
\item \textbf{Scoping}: A name cannot occur outside a binding for it,
  so $x$ is only a valid $\lambda$-term inside some binding for $x$;
  and
%
\item \textbf{Typing}: Each name is associated with exactly one type,
  so e.g.\ $x$ can only be used with type $A$ inside the binding
  $\lamabs{x}{A}{t}$.
\end{itemize}
Note that scoping does not preclude the possibility of open terms. We
allow the possibility that a name is bound outside of a
$\lambda$-term, in the meta-language itself, to allow intermediate
values in a computation that are open. At the top level, though, we
are only really interested in the closed $\lambda$-terms. Thus scoping
is really about \emph{adequacy}.
% It is possible in \ourlib\ to encode
% the $\lambda$-calculus over a given set of free variables, but that
% would be an encoding of a different sort of data\noteP{you mean of something different than $\lambda$-calculus right? I had to stop 5 sec. on that sentence}.

Bindings are encoded in \ourlib\ as pairs of a name \lstinline{n} and
an expression \lstinline{e}, written in this document as
follows\footnote{ $\nu$ is mnemonic for a new name, a usage that seems
  to date to \citet{odersky94}.}:
\[
\nu\text{\lstinline{n.e}}
\]
Intuitively, \lstinline{n} is the name bound by the binding, while
\lstinline{e} is the scope of the binding. Names always have type
\lstinline{Name a} for some type \lstinline{a}.  The $\nu$ is
essentially a constructor, meaning that evaluation is allowed to
proceed inside a binding (up to the lazy evaluation of Haskell).
\ourlib\ guarantees both that \lstinline{n} cannot occur outside of
\lstinline{e} and that \lstinline{n} is always \emph{fresh}, i.e., if
one binding occurs inside another then the two bound names will always
be distinct. In order to achieve these guarantees, bindings are opaque
to the user, meaning that the user cannot see the constructor $\nu$ to
apply it or pattern-match on it. Otherwise the user could build the
term $\nu$\lstinline{n.}$\nu$\lstinline{n.e}, which violates
freshness, or could take the first projection of a binding to extrude
a name from its scope. This means that the $\nu$ construct is actually
pseudo-syntax, used in this paper in order to describe \ourlib\ and
its behavior; in actual programs, the user must use the functions
defined in \ourlib\ to manipulate bindings.

As an example of using \ourlib\ to encode data with bindings, the
simply-typed $\lambda$-terms can be encoded in \ourlib\ using the
following GADT, where \lstinline{Binding a b} is the type of bindings
$\nu$\lstinline{n.e} where \lstinline{n} has type \lstinline{Name a}
and \lstinline{e} has type \lstinline{b}:
\begin{lstlisting}
data Term a where
  Var :: Name a -> Term a
  Lam :: Binding a (Term b) -> Term (a -> b)
  App :: Term (a -> b) -> Term a -> Term b
\end{lstlisting}
Using this type, the simply-typed $\lambda$-terms can be encoded as
follows, where $\Theta$ is a mapping from bound variables to bound
names in \ourlib\ and $\Theta,x\mapsto$\lstinline{n} builds the
obvious extension of $\Theta$:
\[
\begin{array}{lcl}
\lcenc{\Theta}{x} & = & \text{\lstinline{Var}}\;\Theta(x)\\
\lcenc{\Theta}{\lamabs{x}{A}{t}} & = & \text{\lstinline{Lam}}\;(\nu\text{\lstinline{n.}}\lcenc{\Theta,x\mapsto \text{\lstinline{n}}}{t})\\
\lcenc{\Theta}{t\;u} & = & \text{\lstinline{App}}\;\lcenc{\Theta}{t}\;\lcenc{\Theta}{u}
\end{array}
\]
For example, the $\lambda$-term $\lamabs{s}{A\to A}{\lamabs{z}{A}{s\;z}}$
representing the church numeral 1 is encoded as follows:
\begin{lstlisting}[mathescape]
Lam ($\nu$s. Lam ($\nu$z. App (Var s) (Var z)))
\end{lstlisting}

The benefit of this encoding is that it eliminates whole classes of
program bugs. For one, the encoding is \emph{adequate}, meaning here
that the values of type \lstinline[mathescape]{Term $\lcenct{A}$} with
free names of type \lstinline[mathescape]{Name $\lcenct{B_1}$} through
\lstinline[mathescape]{Name $\lcenct{B_n}$} are in bijective
correspondence with the simply-typed lambda-terms of type $A$ with
free variables of type $B_1$ through $B_n$, where $\lcenct{\cdot}$ is
a suitable encoding of lambda-types into Haskell types that maps
function types to function types.\footnote{Note that the bijection
  technically only holds for \emph{strict} values, since the lazy
  evaluation of Haskell allows the creation of infinite terms, meaning
  that \lstinline{Term} can in fact contain representations of
  infinite lambda terms.} This means that values of type
\lstinline{Term a} are \emph{guaranteed} to be well-typed terms, and
thus a program cannot accidentally generate (a representation of) an
ill-formed term. The role that \ourlib\ plays in this guarantee is to
ensure that (representations of) variables are always used in their
correct scopes and at their correct types. Thus, for example, a user
cannot write a function that accidentally removes a variable from its
scope. In addition, due to the Freshness property, a function cannot
accidentally capture a free variable when creating a new binding;
avoiding accidental variable capture often requires complex code to
get right. Finally, the $\alpha$-Equivalence property ensures that all
representations for the same binding are indistinguishable, so the
user cannot accidentally write a function that takes two different
representations of the same term to different values.


%%%%%%%%%%%%%%%%%%%%%%%%%%%%%%%%%%%%%%%%
%%%%%%%%%%     Section        %%%%%%%%%%
%%%%%%%%%%%%%%%%%%%%%%%%%%%%%%%%%%%%%%%%
\section{Using Data with Bindings}
\label{sec:matching-on-bindings}

In this section we show by example how to write operations that
manipulate data with bindings using the primitives of \ourlib. We
first cover some of the more basic primitives. Most of these
primitives take a binding $\nu$\lstinline{n.e} and extract some value
from \lstinline{e} that is guaranteed not to contain \lstinline{n}
free. This is to ensure that the operation does not violate
Scoping. \ourlib\ also contains a primitive to apply certain functions
\lstinline{f} to the body of a binding, that is, to take binding
$\nu$\lstinline{n.e} and return $\nu$\lstinline{n.(f e)}. Finally,
Section \ref{subsec:matching-under-bindings} shows how to
pattern-match on the bodies of bindings by commuting constructors out
of bindings.


In order to make pattern-matching (discussed below) easier to use, we
generalize bindings to \emph{multi-bindings} which bind zero or more
fresh names. We write multi-bindings on paper as follows:
\[
\nu\text{\lstinline{(n1, ..., nk).e}}
\]
Similar to bindings, this construct is a pair of a list of fresh names
and a body \lstinline{e} in which those names are bound, where
$\nu$ is again an opaque constructor. The type of such a multi-binding
is
\[
\text{\lstinline{Mb (CtxCons ... (CtxCons CtxNil a1) ... ak) b}}
\]
where \lstinline{n1}, \ldots, \lstinline{nk} are names of type
\lstinline{Name a1}, \ldots, \lstinline{Name ak}, respectively, and
\lstinline{e} has type \lstinline{b}.  Note that the types are listed
\emph{inside-out}, meaning that the type of the name bound last in the
multi-binding is listed first. This convention is useful in the common
case, when more bindings are added to the inside of a multi-binding.
This will become more clear in the examples below.  The types
\lstinline{CtxNil} and \lstinline{CtxCons l a} are so-called
\emph{phantom types} \cite{leijen99}; i.e., they are only used to
represent lists of types at the type level, and do not represent any
data.


Figure \ref{fig:hobbit-ops} summarizes the external interface of
\ourlib, other than the pattern-matching facility discussed in Section
\ref{subsec:matching-under-bindings}. This figure begins with a number
of ``helper'' declarations.\footnote{The declarations involving
  type contexts are actually separated into the companion
  \lstinline{Ctx} module, also bundled with \ourlib.}
The first two lines declare the phantom types
\lstinline{CtxNil} and \lstinline{CtxCons} for constructing type-level
lists. The next declaration defines \lstinline{Binding}s as
multi-bindings that bind just one name.  After this is a declaration
for the type
% comment to prevent meta-q from wrapping the \lstinline
\lstinline{a :=: b} of \emph{proof objects} which prove that type
\lstinline{a} equals type \lstinline{b}. This is a GADT with one
constructor, \lstinline{Refl}, which proves that \lstinline{a :=: a}
for any type \lstinline{a}.
% (See, e.g., [FIXME] for a more in-depth
% discussion of proof objects in Haskell.)
Since \lstinline{Refl} is the only constructor for
% comment to prevent meta-q from wrapping the \lstinline
\lstinline{a :=: b}, it is straightforward to see that a value of type
\lstinline{a :=: b} can only be constructed if type \lstinline{a} and
\lstinline{b} are in fact the same types.


\begin{figure}[t]
\vspace{-5pt}
\begin{lstlisting}
-- phantom types
data CtxNil
data CtxCons l a

-- bindings are one-name multi-bindings
type Binding a b = Mb (CtxCons CtxNil a) b

-- proofs of type equality
data a :=: b where Refl :: a :=: a

-- proofs that a type is in a context
data InCtx ctx a where
  InCtxBase :: InCtx (CtxCons ctx a) a
  InCtxStep :: InCtx ctx a ->
               InCtx (CtxCons ctx b) a

-- context append
type family (ctx1 :++: ctx2)
type instance ctx1 :++: CtxNil = ctx1
type instance ctx1 :++: (CtxCons ctx2 a) =
  (CtxCons (ctx1 :++: ctx2) a)

-- proofs of context append
data IsAppend ctx1 ctx2 ctx where
  IsAppendBase :: IsAppend ctx CtxNil ctx
  IsAppendStep ::
    IsAppend ctx1 ctx2 ctx ->
    IsAppend ctx1 (CtxCons ctx2 x)
                  (CtxCons ctx x)

-- operations of HobbitLib
nu :: (Name a -> b) -> (Binding a b)
emptyMb :: a -> Mb CtxNil a
elimEmptyMb :: Mb CtxNil a -> a

combineMb :: Mb ctx1 (Mb ctx2 a) ->
             Mb (ctx1 :++: ctx2) a

separateMb :: IsAppend ctx1 ctx2 ctx ->
              Mb ctx b -> Mb ctx1 (Mb ctx2 b)

cmpName :: Name a -> Name b -> Maybe (a :=: b)

mbNameBoundP :: Mb ctx (Name a) ->
                Either (InCtx ctx a) (Name a)

mbToplevel :: SuperComb (a -> b) ->
              Mb ctx a -> Mb ctx b
\end{lstlisting}
\caption{External Interface to \ourlib}
\label{fig:hobbit-ops}
\end{figure}


The next declaration in Figure \ref{fig:hobbit-ops} defines the type
\lstinline{InCtx ctx a}.  This GADT defines \emph{proof objects} that
witness the fact that the type \lstinline{a} is in the type context
\lstinline{ctx}. The first constructor, \lstinline{InCtxBase}, is a
proof that the last type in a type context is in that type
context. The second constructor, \lstinline{InCtxStep}, takes a proof
that type \lstinline{a} is in context \lstinline{ctx} and builds a
proof that \lstinline{a} is in the result of adding any other type to
the end of \lstinline{ctx}.  Given these declarations, it is
straightforward to see that, given a value of type
\lstinline{InCtx ctx a}, it must be the case that \lstinline{a} is
listed somewhere in \lstinline{ctx}.


The last helper declarations in Figure \ref{fig:hobbit-ops} define an
append operation \lstinline{ctx1 :++: ctx2} on type contexts
\lstinline{ctx1} and \lstinline{ctx2}.  This operation uses the
Haskell type family functionality that allows for the definition of
computations on types. The first \lstinline{instance} declaration
states that the type \lstinline{ctx1 :++: CtxNil} is equal to the type
\lstinline{ctx1}, while the second states that
\lstinline{ctx1 :++: (CtxCons ctx2 a)} equals the result of appending
\lstinline{ctx2} to the end of \lstinline{ctx1} and then appending
\lstinline{a}. Figure \ref{fig:hobbit-ops} also defines the type
\lstinline{IsAppend ctx1 ctx2 ctx} of proof objects that witness the
fact that appending \lstinline{ctx1} and \lstinline{ctx2} yields
\lstinline{ctx}. The definition of this type mirrors the definition of
the \lstinline{:++:} type function.


After these helper declarations, Figure \ref{fig:hobbit-ops} gives the
binding-related operations of \ourlib.  First is the \lstinline{nu}
operator, which is the way to create bindings in \ourlib.
Intuitively, this operator creates a binding by generating a fresh
name and passing it to a user-supplied function \lstinline{f}, which
returns the body of the newly created binding.  More technically,
\lstinline{nu f} generates a fresh name \lstinline{n} and then reduces
to $\nu$\lstinline{n. f n}, where the reduction of \lstinline{f n} can
then occur inside the binding. For example, the code
\begin{lstlisting}
Lam (nu (\s ->
  Lam (nu (\z -> App (Var s) (Var z)))))
\end{lstlisting}
creates the church numeral 1 given above.  Note that ``generates a
fresh name'' sounds at first like it violates Haskell's purity
restriction that allows only side-effect-free code. Indeed, inside the
\ourlib\ library, \lstinline{nu} uses the infamous
\lstinline{unsafePerformIO} operation. We show in Section
\ref{sec:safety}, however, that \lstinline{nu} is \emph{effectively}
pure, meaning that its side effects cannot be observed (except inside
\ourlib).


After \lstinline{nu}, Figure \ref{fig:hobbit-ops} declares
\lstinline{emptyMb} and \lstinline{elimEmptyMb}. These intuitively add
and remove, respectively, an empty multi-binding with no names around
a value. Thus, for example, if \lstinline{i} is an integer then
\lstinline{emptyMb i} intuitively returns the result
$\nu$\lstinline{().i}, and applying \lstinline{elimEmptyMb} to this
result removes the empty name list and returns \lstinline{i} again.


Next comes the \lstinline{combineMb} operation, which takes two nested
multi-bindings and combines them into a single multi-binding
containing the names of both multi-bindings. The type of
\lstinline{combineMb} expresses that it takes a multi-binding for
\lstinline{ctx1} that contains a nested multi-binding for
\lstinline{ctx2} and returns a multi-binding for
% comment to prevent meta-q from wrapping the \lstinline
\lstinline{ctx1 :++: ctx2}.  The inverse operation,
\lstinline{separateMb}, takes a multi-binding for
\lstinline{ctx1 :++: ctx2} and separates it into two nested
multi-bindings for \lstinline{ctx1} and \lstinline{ctx2},
respectively. For any given type context, however, there may be multiple ways
to represent it as a context append \lstinline{ctx1 :++: ctx2}. The first
argument to \lstinline{separateMb}, therefore, both guarantee that the context
can be divided in the requested way and also inform the Haskell type system
which \lstinline{ctx1} and \lstinline{ctx2} are intended.


Names can be compared using the \lstinline{cmpName} operator, which
takes any two names, possibly of different types, and compares them.
If the names are equal, then \lstinline{cmpName} returns
\lstinline{Just p}, where \lstinline{p} is a proof that the types of
the two names are equal. This is because \ourlib\ dictates that a name
must have exactly one type, by the Typing property. If two names are
not equal, then \lstinline{cmpName} returns the value
\lstinline{Nothing}.

The programmer can also test if a name inside a multi-binding is bound
by that multi-binding using the \lstinline{mbNameBoundP} operation.
This operation takes an input of type \lstinline{Mb ctx (Name a)} and
returns either the name itself, if the name is not bound by the
multi-binding, or a result of type \lstinline{InCtx ctx a}.
Intuitively, the latter type is used because, if a name is bound by a
multi-binding of names whose types are listed in the type context
\lstinline{ctx}, then its type must in \lstinline{ctx}.


%%%%%%%%%%%%%%%%%%%%%%%
%%%%% Sub-Section %%%%%
%%%%%%%%%%%%%%%%%%%%%%%
\subsection{Operating under Bindings}
\label{subsec:mbtoplevel}

The final operation listed in Figure \ref{fig:hobbit-ops} is
\lstinline{mbToplevel}, which allows functions to be applied inside
bindings. This operation takes a function \lstinline{f} and a
multi-binding
% comment here so meta-q does not wrap the \lstinline...
$\nu$\lstinline{(n1, ..., nk).e} and returns the result of
$\nu$\lstinline{(n1, ..., nk).(f e)}. This operation is only safe,
however, when \lstinline{f} has no free names; otherwise, the danger
is that \lstinline{f} could contain some other binding for some
\lstinline{ni}. This would violate Freshness, since the result would
contain a binding for \lstinline{ni} inside another binding for the
same name. Stated differently, since \ourlib\ cannot in general
perform $\alpha$-conversion on bindings, it instead disallows cases
where $\alpha$-conversion is required.

One sufficient condition for guaranteeing that the \lstinline{f}
argument to \lstinline{mbToplevel} has no free names is to restrict it
to \emph{super combinators}, i.e., expressions that are top-level,
whose only free variables reference top-level definitions. Since the
\ourlib\ interface does not permit the top-level definition of a
\lstinline{Name} value (because of Scoping), this prevents
\lstinline{mbToplevel} from violating Freshness.

To ensure that functions passed to \lstinline{mbToplevel} are super
combinators, \ourlib\ uses Haskell's metaprogramming facility,
Template Haskell \cite{sheard-peyton-jones-02}. Specifically, \ourlib\
defines a type \lstinline{SuperComb} for representing super
combinators, along with a function \lstinline{superComb} for
creating elements of this type. The \lstinline{superComb} function
takes a quoted Template Haskell expression and validates that
the expression is indeed a super combinator, raising an error
if not. In addition, \ourlib\ leaves the \lstinline{SuperComb} type
abstract, ensuring that any element of this type created by
the user must be validated by \lstinline{superComb}.

As an example, we demonstrate how to use \lstinline{superComb} and
\lstinline{mbTopLevel} to define a function that adds a binding to the
inside of a multi-binding. This function is called
\lstinline{mbLower}, as it intuitively \emph{lowers} the body of a
multi-binding into a context with more bindings, and can be defined as
follows:
\begin{lstlisting}
mbLower :: Mb ctx a -> Mb (CtxCons ctx b) a
mbLower =
  combineMb .
  mbToplevel $(superComb [| nu . const |])
\end{lstlisting} %% $
The \lstinline{mbLower} function works by first creating a quoted
Template Haskell expression using the Template Haskell brackets
(\lstinline{[|} and \lstinline{|]}), applying \lstinline{superComb} to
validate the syntactic restrictions for super combinators, and then
using the Template Haskell \lstinline{$} operator to splice the
wrapped quotation back in as the first argument of
\lstinline{mbToplevel}. Note that this pattern of usage ensures that
\lstinline{superComb} is called at compile time, since it occurs
inside the \lstinline{$} operator, and thus any errors raised by
\lstinline{superComb} --- because the argument is not a valid super
combinator --- are signaled before the code is run.

The quoted expression here composes \lstinline{const} and
\lstinline{nu}.  The first of these, \lstinline{const}, takes an
expression \lstinline{e} and builds a function that returns
\lstinline{e}, ignoring its argument, while \lstinline{nu} then
generates a fresh name and passes the name to the newly created
function, thereby creating a binding whose body is \lstinline{e}.  The
use of \lstinline{mbToplevel} then applies this function inside a
multi-binding, creating a new binding inside the existing
multi-binding of its argument. Finally, \lstinline{combineMb} combines
these two bindings to create a single, unified multi-binding.

% \begin{figure}
% \begin{lstlisting}[basicstyle=\ttfamily\scriptsize]
% mbLower :: Mb ctx a -> Mb (CtxCons ctx b) a
% mbLower = combineMb . mbToplevel $(superComb [| nu . const |])
% \end{lstlisting}
% \caption{Example Operation \lstinline{mbLower} Defined with \lstinline{superComb}}
% \label{fig:mbtoplevel-examples}
% \end{figure}


% \begin{figure}
% \begin{lstlisting}[basicstyle=\ttfamily\scriptsize]
% -- empty data declaration for tagging purposes
% data MbMbNBP

% -- instance with topFun = mbNameBoundP
% instance ToplevelFun MbMbNBP
%                      (Mb ctx (Name a)) where
%   type ToplevelRes MbMbNBP (Mb ctx (Name a)) =
%     Either (InCtx ctx a) (Name a)
%   topFun _ n = mbNameBoundP n

% -- using mbNameBoundP
% mbMbNameBoundP
%     :: Mb ctx1 (Mb ctx2 (Name a)) ->
%        Mb ctx1 (Either (InCtx ctx2 a) (Name a))
% mbMbNameBoundP = mbToplevel (Tag :: Tag MbMbNBP)

% -- testing if a name is the last one bound
% cmpLastName :: Mb (CtxCons ctx b) (Name a) ->
%                 Maybe (Mb ctx (Name a))
% cmpLastName n =
%   case mbMbNameBoundP
%          (separateMb
%           (IsAppendStep IsAppendBase) n) of
%     [nuQQ| Left _ |] -> Nothing
%     [nuQQ| Right n |] -> Just [nuQQ| n |]

% -- adding a binding inside another
% data MbLowerRTag b
% instance ToplevelFun (MbLowerRTag b) a where
%     type ToplevelRes (MbLowerRTag b) a =
%         Binding b a
%     topFun _ x = nu $ const x

% mbLower :: Mb ctx a -> Mb (CtxCons ctx b) a
% mbLower = combineMb . (mbToplevel
%                         (Tag :: Tag (MbLowerRTag b)))

% -- a reification of a context at the term level
% type Ctx ctx = MapCtx Tag ctx

% -- lowering into multiple bindings
% lowerMulti :: Ctx ctx -> a -> Mb ctx a
% lowerMulti EmptyMC a = emptyMb a
% lowerMulti (ctx :> Tag) a =
%   combineMb $ lowerMulti ctx (nu (\_ -> a))

% -- make a bound name for the last name in the context
% mkBoundName :: Ctx ctx -> Mb (CtxCons ctx a) (Name a)
% mkBoundName ctx =
%   combineMb (lowerMulti ctx (nu (\n -> n)))
% \end{lstlisting}
% \caption{Example Operations Defined with \lstinline{mbToplevel}}
% \label{fig:mbtoplevel-examples}
% \end{figure}


%%%%%%%%%%%%%%%%%%%%%%%
%%%%% Sub-Section %%%%%
%%%%%%%%%%%%%%%%%%%%%%%
\subsection{Pattern-Matching under Bindings}
\label{subsec:matching-under-bindings}

The most powerful facilities of \ourlib\ are those allowing for
pattern-matching on the contents of multi-bindings. Specifically,
\ourlib\ allows the user to match terms of type \lstinline{Mb ctx a}
against patterns $\nu$\lstinline{(n1,...,nk).P} where \lstinline{P} is
a pattern of type \lstinline{a}.  To support this, we must solve two
issues: we do not wish to expose the $\nu$ constructor to the
programmer in order to ensure Freshness and Scoping, as discussed
above; and, even without exposing the $\nu$ constructor directly, the
most straightforward approach to pattern-matching under bindings can
accidentally violate Scoping. We discuss these two problems and our
solution in the remainder of this section.


To solve the first issue, that we do not wish to expose the $\nu$
constructor to the user, we use Haskell's quasi-quotation mechanism
\cite{mainland07} to allow \ourlib\ to automatically and safely
generate Haskell patterns to match multi-bindings. If \lstinline{P} is
a Haskell pattern, then the programmer can write
% spacing for \lstinline
\lstinline{[nuQQ| P |]} to match a multi-binding whose body matches
\lstinline{P}. We call such patterns $\nu$-patterns in the below.
Internally, this syntax directs the Haskell compiler to call the
quasi-quoter \lstinline{nuQQ}, defined in the \ourlib\ library, at
compile time to generate the necessary pattern. As an example, the
following defines a function \lstinline{mbBool} that intuitively lifts
a boolean value out of a multi-binding:
\begin{lstlisting}
mbBool :: Mb ctx Bool -> Bool
mbBool [nuQQ| True |] = True
mbBool [nuQQ| False |] = False
\end{lstlisting}
The first case matches $\nu$\lstinline{(n1,...,nk).True} for any list
of names \lstinline{n1} through \lstinline{nk}. It then gives
\lstinline{True}, outside of the multi-binding, as the return
value. Similarly, the second case matches a multi-binding containing
\lstinline{False} and returns \lstinline{False} outside of the
multi-binding. We use the term \emph{lifting function} for functions
like \lstinline{mbBool} which intuitively lift a value out of a
multi-binding.


This approach is not possible for types that effectively have
infinitely many constructors, such as the integers. For such types,
\ourlib\ exports special-purpose lifting functions; at present,
these include the following:
\begin{lstlisting}
mbInt :: Mb ctx Int -> Int
mbChar :: Mb ctx Char -> Char
mbString :: Mb ctx String -> String
\end{lstlisting}
Note that the last function, \lstinline{mbString}, can in fact be
defined by the user using \lstinline{mbChar} and pattern-matching over
lists, but this operation is needed often enough that we include it in
\ourlib. In addition, a user-defined \lstinline{mbString} function
would in fact take time linear in the size of the string, while the
internally defined version is constant time. These internally-defined
lifting functions also behave slightly differently than user-defined
lifting functions with regards to termination, as the latter require
the body of a multi-binding to have a weak head normal form. This is
not usually a problem in practice, however.


The second issue that arises in defining pattern-matching under
bindings is that the most straightforward approach can lead to a
violation of Scoping. This has to do with how variables are handled in
$\nu$-patterns. Consider, for example, the following function:
\begin{lstlisting}
unsafe :: Mb ctx (Name a) -> Name a
unsafe [nuQQ| x |] = x
\end{lstlisting}
Intuitively, this function matches any name inside a multi-binding,
including names that are bound by the multi-binding itself, and
removes that name from the multi-binding. Note that such a function
can also be unsafe for types other than \lstinline{Name a} such as
the \lstinline{Term a} type defined above that can contain names.

To address this issue, the pattern generated by the
% comment here to prevent meta-q from wrapping the \lstinline
\lstinline{[nuQQ| P |]} quasiquotation implicitly ``re-wraps'' the
multi-binding of the argument around all variables bound by
\lstinline{P}. Thus pattern variables from $\nu$-patterns are never
separated from the context where their names are guaranteed to be
bound, and so Scoping is preserved. The \lstinline{unsafe} function
above is actually ill-typed; the type of \lstinline{x} is again
\lstinline{Mb ctx (Name a)}, so changing the type signature
accordingly would essentially yield the identity function. The
\lstinline{nuQQ} quasi-quoter is defined in Section
\ref{sec:implementation}.

As an example, the following computes the length of a list
inside a multi-binding:
\begin{lstlisting}
mbLen :: Mb ctx [a] -> Int
mbLen [nuQQ| [] |] = 0
mbLen [nuQQ| x:l |] = 1 + mbLen l
\end{lstlisting}
The first case matches the empty list inside a multi-binding and
returns 0. The second case matches a cons \lstinline{x:l} inside a
multi-binding and returns 1 plus the result of the recursive call on
\lstinline{l}. Recall that \lstinline{l} has type % (for lstinline)
\lstinline{Mb ctx [a]} on the right-hand side; the \lstinline{nuQQ}
quasi-quoter has already re-wrapped the tail of the list in the
multi-binding to preserve Scoping. Note again that allowing the tail
of the list to escape the scope of its multi-binding could violate
Scoping if, for example, it is a list of names, but this might not be
known until run-time as the type of \lstinline{mbLen} is polymorphic
in \lstinline{a}.

%%%%%%%%%%%%%%%%%%%%%%%
%%%%% Sub-Section %%%%%
%%%%%%%%%%%%%%%%%%%%%%%
\subsection{Illustrative Examples}
\label{subsec:small-examples}

In this section, we illustrate the use of \ourlib\ by giving two small
examples using the \lstinline{Term} type of simply-typed lambda-terms
defined above. A larger example is given in Section
\ref{sec:lambda-lifting}. The examples are listed in Figure
\ref{fig:small-examples}, along with some helper declarations which
shall be described below. These include a function \lstinline{eq} that
tests equality of two terms and a function \lstinline{subst} which
performs capture-avoiding, type-safe substitution of a term into a
binding. Note that both of these are defined using helper functions,
\lstinline{meq} and \lstinline{msubst}, respectively, that operate on
a term in a multi-binding.  This pattern of defining a function on
type \lstinline{a} by defining a function over multi-bindings
containing \lstinline{a} is common in \ourlib. At a high level, this
approach is essentially avoiding $\alpha$-conversion by maintaining a
context of all the bound names that have been seen during the
traversal. Note that this is different from approaches such as Beluga
\cite{pientka08} where \emph{all} the bound names must be listed with
the term; instead, in \ourlib, there is always the possibility that a
name is bound outside the current scope in which the current function
is being called.
% Because we avoid $\alpha$-conversion, this also means that \ourlib\ is
% efficient (see Section \ref{sec:performance}), since it need not
% $\alpha$-convert the whole subterm each time it traverses a binder.


\begin{figure}[t]
\vspace{-5pt}
\begin{lstlisting}
-- test if two InCtx proofs have
-- the same length (code omitted)
inCtxSameLen :: InCtx cxt1 a1 ->
                InCtx cxt2 a2 -> Bool

-- boolean version of cmpName (code omitted)
cmpNameBool :: Name a -> Name b -> Bool

-- equality under multi-bindings
meq :: (Mb ctx1 (Term a1)) ->
       (Mb ctx2 (Term a2)) -> Bool
meq [nuQQ| Var bv1 |] [nuQQ| Var bv2 |] =
  case (mbNameBoundP bv1, mbNameBoundP bv2) of 
    (Left p1, Left p2)   -> inCtxSameLen p1 p2
    (Right n1, Right n2) -> cmpNameBool n1 n2
    _                    -> False
meq [nuQQ| App f1 a1 |] [nuQQ| App f2 a2 |] =
  meq f1 f2 && meq a1 a2
meq [nuQQ| Lam b1 |] [nuQQ| Lam b2 |] =
  meq (combineMb b1) (combineMb b2)
meq _ _ = False

-- equality of terms
eq :: Term a -> Term b -> Bool
eq t u = meq (emptyMb t) (emptyMb u)

-- tuples whose types are indexed by ctx
data MapCtx f ctx where
  Empty :: MapCtx f CtxNil
  (:>) :: MapCtx f ctx -> f a ->
          MapCtx f (CtxCons ctx a)

-- tuple lookup (code omitted)
ctxLookup :: InCtx ctx a -> MapCtx f ctx -> f a

-- multi-arity substitution
msubst :: Mb ctx (Term a) ->
          MapCtx Term ctx -> Term a
msubst [nuQQ| Var bn |] ts =
  case mbNameBoundP bn of
    Left p  -> ctxLookup p ts
    Right n -> Var n
msubst [nuQQ| App f a |] ts =
  App (msubst f ts) (msubst a ts)
msubst [nuQQ| Lam b |] ts =
  Lam $ nu $ \n ->
    msubst (combineMb b) (ts :> Var n)

-- substituting a single term
subst :: Binding a (Term b) -> Term a -> Term b
subst b t = msubst (Empty :> t) b
\end{lstlisting}
\caption{Equality and Substitution in \ourlib}
\label{fig:small-examples}
\end{figure}


The \lstinline{meq} function takes two terms in multi-bindings and
tests if the terms are equal. Note that it does not require the two
terms to have the same type or the multi-bindings to have the same
type contexts, as this would make it much more complex to write the
case for \lstinline{App}. The first case matches terms that are both
variables, i.e., that both use the \lstinline{Var} constructor. It
then uses \lstinline{mbNameBoundP} to determine if both variables are
bound in their respective multi-bindings: if so, then
\lstinline{inCtxSameLen} (whose code is omitted) is called on the
\lstinline{InCtx} proofs returned, to determine if both names were
bound at the same place in the term; if both names are free then they
are compared with \lstinline{cmpNameBool} (whose code is omitted);
otherwise, if one is bound and the other is free, they cannot be
equal, and \lstinline{False} is returned.


The second case of \lstinline{meq} matches two applications, i.e., two
terms in multi-bindings that both start with \lstinline{App}.  This
case recurses on the two pairs of subterms, checking if the two
functions and arguments are equal.  The third case, where both terms
start with \lstinline{Lam}, is similar, except that arguments to
\lstinline{Lam} are themselves bindings, so the results of the
$\nu$-patterns are bindings inside multi-bindings. Thus
\lstinline{combineMb} is called to combine the inner and outer
bindings, resulting in a pair of terms inside larger multi-bindings,
and the recursion then proceeds on the results.  This case illustrates
why type contexts are represented inside-out, as discussed in the
beginning of Section \ref{sec:matching-on-bindings}, since
$\nu$-patterns for binding constructs like \lstinline{Lam} result in
single bindings inside of multi-bindings.  The final case matches
terms with different constructors, in which case \lstinline{False} is
returned. The \lstinline{eq} function, for comparing terms that are
not in multi-bindings, is then defined by first calling
\lstinline{emptyMb} on both terms to put them inside empty
multi-bindings, and then by calling \lstinline{meq} on the results.


The second example in Figure \ref{fig:small-examples} defines the
function \lstinline{msubst}, which performs multi-arity substitution.
The first argument to \lstinline{msubst} is a term of type
\lstinline{Term a} inside a multi-binding with type context
\lstinline{ctx}. The second argument is a tuple containing one
lambda-term for each type listed in \lstinline{ctx}, i.e., containing
a lambda-term of type \lstinline{b} for each name of type
\lstinline{b} bound by the multi-binding in the first argument.
Intuitively, these terms are being substituted for the bound names in
the first argument. The type of the second argument is
\lstinline{MapCtx Term ctx}, which intuitively maps the type function
\lstinline{Term} over the types listed in \lstinline{ctx} using the
GADT given in the figure.  This tuple argument is used by the first
case of \lstinline{msubst}, which matches the \lstinline{Var} case of
the first argument.  In this case, \lstinline{mbNameBoundP} is called
to determine if the name used by \lstinline{Var} is bound in the
multi-binding around the term: if so, then \lstinline{ctxLookup}
(whose code is omitted) is called to return the term in the tuple
argument that has the same type as the bound name; otherwise, the name
is free and is removed from the multi-binding, and so is returned as
the result.  The second case of \lstinline{msubst}, for
\lstinline{App}, simply recurses on the two subterms.

The third case, for \lstinline{Lam} recurses on the result of calling
\lstinline{combineMb} on the argument to \lstinline{Lam}, as in
\lstinline{meq}. The result is put inside a new application of
\lstinline{Lam}, which uses \lstinline{nu} to bind a new name
\lstinline{n}. Note that we cannot re-use the old name that was
previously bound at this location, because of the possibility that one
of the terms being substituted into the body of the \lstinline{Lam}
(i.e., one of the terms listed in the second argument) uses the same
name. Thus this sort of re-use could potentially violate
Freshness. The second argument, which is the tuple of terms to use for
bound names, is then extended to contain the new bound name
\lstinline{n}. This explains why substitution is defined here as
multi-arity substitution, in order to ``freshen up'' the bound names.
Note that it is possible instead to define an $\alpha$-conversion
operation in a similar manner, using a tuple argument of type
\lstinline{MapCtx Name ctx} to contain the fresh names to be used in
place of the old ones, but combining the operation of freshening up
the names with the definition of substitution yields a definition of
substitution that is quadratic instead of cubic.  The function
\lstinline{subst}, which substitutes a single term into another, is
then defined with a call to \lstinline{msubst} using a one-element
\lstinline{MapCtx} tuple.


%%%%%%%%%%%%%%%%%%%%%%%%%%%%%%%%%%%%%%%%
%%%%%%%%%%     Section        %%%%%%%%%%
%%%%%%%%%%%%%%%%%%%%%%%%%%%%%%%%%%%%%%%%
\section{Extended Example: Lambda-Lifting}
\label{sec:lambda-lifting}

In this section, we demonstrate how \ourlib\ can be used on a larger
example, lambda-lifting (also known as closure conversion; see, e.g.,
\cite{peyton-jones87}). Using \ourlib, our approach defines
lambda-lifting as a function over an adequate encoding of simply-typed
lambda-terms with top-level declarations. This means that our
lambda-lifting function is \emph{statically guaranteed} by the Haskell
compiler to be type-preserving, meaning that it can only produce an
output of the same type as the input. The code described here has been
implemented, type-checked, and tested using GHC version 7.0.3, and
takes 170 lines of non-comment, non-whitespace Haskell (not including
tests, pretty-printing, and extensions). It can be downloaded with
\ourlib\ at \lstinline{http://www.cs.rice.edu/~emw4/hobbits.tgz}


The goal of lambda-lifting is to transform a functional program so
that all functions are top-level declarations, thus making it easier
to compile. To do this, each lambda-abstraction that occurs in a term
is ``lifted'' to be a top-level declaration. The process of lifting a
lambda also abstracts over all variables that occur free in that
lambda, since those variables will not be bound at the top level. The
original occurrence of the lambda is then replaced by a call to this
new top-level declaration, applying it to all the variables that were
free in the original lambda.

As an example, consider the following term:
\[
\lamabsnot{f}{\lamabsnot{g}{f\;(\lamabsnot{x}{g\;(g\;x)})}}
\]
To lambda-lift this term, we lift the inner lambda (the argument to
$f$) to be a top-level declaration. Since this lambda has the variable
$g$ free, we must abstract over $g$ when we lift it. The occurrence
of the inner lambda then gets replaced by a call to the top-level
declaration applied to the free variable $g$. The entire term is
then lifted as well, since it is itself a lambda, and the result
is as follows, where we use \letname\ for top-level declarations
and use capital letters for the variables they bind:
\[
\lete{F_1}{\lamabsnot{g}{\lamabsnot{x}{g\;(g\;x)}}}{
\lete{F_2}{\lamabsnot{f}{\lamabsnot{g}{f\;(F_1\;g)}}}{
  F_2
}}
\]
Note that the declaration for $F_1$, which corresponds to the inner
lambda, has an extra lambda-abstraction for the variable $g$. The
original occurrence of the inner lambda has been replaced by a call to
$F_1$, which is applied to the variable $g$.


Figure \ref{fig:decls} defines two types derived from the
\lstinline{Term} type given in Section~\ref{sec:data-with-bindings},
one for the source terms to be lambda-lifted and one for the resulting
terms. In these new types, bound names are separated into two sorts by
their types: names of type \lstinline{D a} (for some \lstinline{a})
are bound by declarations, while those of type \lstinline{L a} are
bound by lambda-abstractions of the parameter list of a
declaration. The source type, again called \lstinline{Term}, merely
adds the type \lstinline{L} to the bindings and variable occurrences
in the original \lstinline{Term} type. Because lambda-lifting replaces
all lambdas with top-level declarations, the result type, which we
call \lstinline{DTerm}, has no lambda constructor. It also has
separate constructors \lstinline{Var} and \lstinline{DVar} for
occurrences of lambda-bound and declaration-bound variables,
respectively.


\begin{figure}
\vspace{-5pt}
\begin{lstlisting}[basicstyle=\ttfamily\scriptsize]
-- type tags for declaration- vs lambda-bound names
data D a
data L a

-- terms with only lambda-bound variables
data Term :: * -> * where
  Var :: Name (L a) -> Term a
  Lam :: Binding (L b) (Term a) -> Term (b -> a)
  App :: Term (b -> a) -> Term b -> Term a

-- terms under top-level declarations (no lambdas)
data DTerm a where
  TVar :: Name (L a) -> DTerm a
  TDVar :: Name (D a) -> DTerm a
  TApp :: DTerm (a -> b) -> DTerm a -> DTerm b

-- top-level declarations have at least one parameter
data Decl a where
  DeclBase :: Binding (L b) (DTerm a) -> Decl (b -> a)
  DeclCons :: Binding (L b) (Decl a) -> Decl (b -> a)

-- top-level declarations paired with a return value
data Decls a where
  DeclsBase :: DTerm a -> Decls a
  DeclsCons :: Decl b -> Binding (D b) (Decls a) ->
               Decls a
\end{lstlisting}
\caption{Terms inside Top-level Declarations}
\label{fig:decls}
\end{figure}



Figure \ref{fig:decls} also defines both the type \lstinline{Decl a},
which represents the non-empty parameter list and body of a top-level
declaration, and the type \lstinline{Decls a}, which represents a term
of type \lstinline{a} inside a list of zero or more top-level
declarations. The \lstinline{DeclBase} and \lstinline{DeclCons}
constructors re-use the \lstinline{L} type, since each binding
corresponds to a lambda. Both constructors' types are analogous to
that of the \lstinline{Lam} constructor, with the important
distinction that the \lstinline{DeclBase} constructor maps a term
inside a binding to a declaration. The \lstinline{DeclsBase}
constructor represents a term with no declarations, while
\lstinline{DeclsCons} adds a declaration of some type
\lstinline{b}. This constructor takes a term of type \lstinline{b},
representing the value of the declaration, along with a
\lstinline{Decls a} inside a binding that binds a name for the
declaration. The bound name has type \lstinline{D b} to indicate it is
a declaration variable. As an example, the result of the
lambda-lifting shown above is represented as follows:
\begin{lstlisting}[mathescape]
DeclsCons
  (DeclCons $\nu$g. DeclBase $\nu$x.
     App (Var g) (App (Var g) (Var x)))
  ($\nu$F1.
     DeclsCons
       (DeclCons $\nu$f. DeclBase $\nu$g.
          (App (Var f)
               (App (DVar F1) (Var g)))))
       ($\nu$F2. DeclsBase (DVar F2))
\end{lstlisting}



We present our approach to lambda-lifting in the remainder of this
section. Section \ref{subsec:free-vars} first introduces a data type
of lists of variables which occur free in terms, and briefly describes
a subset relation and a union operation on these lists. These are
needed to describe the free variables of the body of a
lambda-abstraction in a statically-checkable manner. Section
\ref{subsec:term-skeletons} next introduces a type of \emph{term
  skeletons}, which are intermediate representations of the bodies of
lambda-abstractions after lifting. Finally, Section
\ref{subsec:lambda-lifting} gives the main function
\lstinline{llBody}, which makes essential use of the continuation
monad to move computations inside top-level declaration bindings.


% %%%%%%%%%%%%%%%%%%%%%%%
% %%%%% Sub-Section %%%%%
% %%%%%%%%%%%%%%%%%%%%%%%
% \subsection{Helper Functions: Adding and ``Peeling'' Lambdas}
% \label{subsec:add-peel-lambdas}

% In this section, we briefly describe two helper functions,
% \lstinline{addLambdas} and \lstinline{peelLamdas}, that are used in
% the definition of lambda-lifting. We examine these both because they
% are relatively straightforward, but also because they illustrate, in
% the small, the practice of verified programming in \ourlib.


% The two functions, along with some helper definitions, are given in
% Figure \ref{fig:add-peel-lambdas}. The the figure first defines the
% type \lstinline{IsLType a}. This is the type of proofs that
% \lstinline{a} is an \emph{\lstinline{L}-type}, meaning that it is of
% the form \lstinline{L b} for some \lstinline{b}.  The next type,
% \lstinline{LCtx ctx}, defines the proofs that all types listed in the
% type context \lstinline{ctx} are \lstinline{L}-types. Elements of this
% type, called \lstinline{L}-contexts, are also used below as a
% term-level representation of type contexts of \lstinline{L}-types,
% allowing computation at the term level on type contexts. After this is
% the type function \lstinline{AddArrows ctx b}, which forms the curried
% function type from the \lstinline{L}-types listed in \lstinline{ctx}
% to the return type \lstinline{b}.


% \begin{figure}[t]
% \begin{lstlisting}[basicstyle=\ttfamily\scriptsize]
% -- proof that a type starts with L
% data IsLType a where IsLType :: IsLType (L a)

% -- proof that ctx contains only L types
% type LCtx ctx = MapCtx IsLType ctx

% -- type function for adding arrows to a type
% type family AddArrows ctx b
% type instance AddArrows CtxNil b = b
% type instance AddArrows (CtxCons ctx (L a)) b =
%   AddArrows ctx (a -> b)

% -- adding lambdas around a term (code omitted)
% addLambdas :: LCtx lam_ctx ->
%               (MapCtx Name lam_ctx -> DTerm a) ->
%               DTerm (AddArrows lam_ctx a)
% addLambdas EmptyMC k = k emptyMC
% addLambdas (lam_ctx :> IsLType) k =
%     addLambdas lam_ctx (\vs ->
%         Lam $ nu $ \v -> k (vs :> v))

% -- "peeling" lambdas off a term
% data PeelRet ctx a where
%   PeelRet :: LCtx lam_ctx ->
%              Mb (ctx :++: lam_ctx) (DTerm a) ->
%              PeelRet ctx (AddArrows lam_ctx a)
% peelLambdas :: LCtx lam_ctx ->
%                Mb (ctx :++: lam_ctx) (DTerm a) ->
%                PeelRet ctx (AddArrows lam_ctx a)
% peelLambdas lctx [nuQQ| Lam b |] =
%     peelLambdas (lctx :> IsLType) (combineMb b)
% peelLambdas lctx [nuQQ| b |] = PeelRet lctx b
% \end{lstlisting}
% \caption{Lambda-Lifting: Adding and ``Peeling'' Lambdas}
% \label{fig:add-peel-lambdas}
% \end{figure}


% Using these types, the figure then defines our two functions.  The
% first, \lstinline{addLambdas}, intuitively takes an
% \lstinline{L}-context \lstinline{lam_ctx} along with a
% \lstinline{DTerm a} and adds lambdas around the \lstinline{DTerm a}
% for each type in the \lstinline{L}-context. To allow the
% \lstinline{DTerm a} to depend on the variables bound by the new
% lambdas, this argument is given as a function from lists of names to
% \lstinline{DTerm a}. The result type uses \lstinline{AddArrows} to
% indicate the nested function types of the nested lambdas that are
% added. The function recurses on the \lstinline{L}-context, adding a
% lambda for each type, moving the second argument (which specifies the
% term) inside the binding on each recursive call.


% FIXME HERE:
% Note that \lstinline{addLambdas} has a very strong resemblance to
% continuation-based programming.


% The second function defined in Figure \ref{fig:add-peel-lambdas},
% \lstinline{peelLambdas}, intuitively removes the leading lambdas from
% a term. The function takes an \lstinline{L}-context of the types of
% the lambdas removed so far, along with term being peeled, where the
% latter is in a multi-binding whose last names correspond to the types
% in the \lstinline{L}-context. When the term being peeled is a lambda,
% the function simply recurses, and in the base case, the function
% simply returns the \lstinline{L}-context of lambdas that have been
% removed along with the body of the resulting term. This return type
% requires an existential quantification over the \lstinline{L}-context
% being returned, since this type context cannot be known until the data
% is examined at run time. Representing existential type quantification
% requires introducing a new type, here called
% % spacing for \lstinline
% \lstinline{PeelRet ctx a}, which existentially quantifies over
% \lstinline{L}-context and otherwise captures the pair returned by
% \lstinline{peelLambdas}.


%%%%%%%%%%%%%%%%%%%%%%%
%%%%% Sub-Section %%%%%
%%%%%%%%%%%%%%%%%%%%%%%
\subsection{Free Variable Lists}
\label{subsec:free-vars}

In this section, we explain briefly how our approach represents and
operates on lists of free variables. Figure \ref{fig:free-vars}
defines the type \lstinline{FVList ctx fvs} of lists of names, one for
each of the types in the type context \lstinline{fvs}. (Note that we
use ``type context'' interchangeably with ``list of types'' in the
below.) Each of these has a type of the form \lstinline{L a}, which we
refer to as an \emph{\lstinline{L}-type}, and each name is inside a
multi-binding for \lstinline{ctx}. Intuitively, this type is used for
the lambda-bound variables that occur free in a term being
lambda-lifted. The context \lstinline{ctx} lists the types of the
names bound by lambdas which so far have been traversed by
lambda-lifting. Note that nothing in this type requires the names to
be bound in \lstinline{ctx}, since lambda-lifting can potentially be
applied (using \lstinline{mbToplevel}) inside existing bindings. Also,
as an aside, nothing in this type prevents duplicate free variables.

% \begin{itemize}
% \item lists of free vars are given by \lstinline{FVList ctx fvs} which
%   is defined as \lstinline{MapCtx (MbLName ctx) fvs}

% \item the type \lstinline{SubCtx} defines a (weak) notion of subset on
%   these lists; represented as functions which might throw away names;
%   a stronger notion would parameterize over some f, but not needed

% \item getting the union of two fv lists
%   \begin{itemize}
%   \item give the type \lstinline{FVUnionRet}
%   \item give the signature (not the code!) of \lstinline{fvUnion}
%   \end{itemize}
% \end{itemize}


\begin{figure}[t]
\begin{lstlisting}[basicstyle=\ttfamily\scriptsize]
-- free variable lists and sub-context proofs
data MbLName ctx a where
  MbLName :: Mb ctx (Name (L a)) -> MbLName ctx (L a)
type FVList ctx fvs = MapCtx (MbLName ctx) fvs
type SubCtx ctx' ctx =
  MapCtx Name ctx -> MapCtx Name ctx'

-- the result type for fvUnion
data FVUnionRet ctx fvs1 fvs2 where
  FVUnionRet :: FVList ctx fvs ->
                SubCtx fvs1 fvs -> SubCtx fvs2 fvs ->
                FVUnionRet ctx fvs1 fvs2

-- taking the union of two FVLists (code omitted)
fvUnion :: FVList ctx fvs1 -> FVList ctx fvs2 ->
           FVUnionRet ctx fvs1 fvs2
\end{lstlisting}
\caption{Lambda-Lifting: Free Variable Lists}
\label{fig:free-vars}
\end{figure}


The figure also defines the type \lstinline{SubCtx ctx' ctx}, which
intuitively states that \lstinline{ctx} contains all the types listed
in \lstinline{ctx'}. This is represented by a function that maps names
(not in multi-bindings) whose types are listed in \lstinline{ctx} to
names whose types are listed in \lstinline{ctx'}, intuitively dropping
the names whose types occur in \lstinline{ctx} but not
\lstinline{ctx'}. Note that this type is not a ``proof'' in a strict
sense, since such functions could conceivably add names to
\lstinline{ctx'}, but the type is strong enough to suit our purposes
here.


Finally, Figure \ref{fig:free-vars} declares the function
\lstinline{fvUnion} (code omitted) to take the union of two free
variable lists.  The result type states that there is some type
context \lstinline{fvs} such that the two input contexts
\lstinline{fvs1} and \lstinline{fvs2} are sub-contexts of it. Because
of the existential, stating this type requires introducing a new
GADT, \lstinline{FVUnionRet}.



%%%%%%%%%%%%%%%%%%%%%%%
%%%%% Sub-Section %%%%%
%%%%%%%%%%%%%%%%%%%%%%%
\subsection{Term Skeletons}
\label{subsec:term-skeletons}

In order to abstract over all the free variables in a term, our
approach converts the body of each lambda-abstraction to a pair of a
free variable list \lstinline{fvs} and an intermediate representation
of the body that is guaranteed to have only the variables listed in
\lstinline{fvs} free. This is a closedness condition; unfortunately,
\ourlib\ cannot directly represent closedness, since a name in a
multi-binding need not always be bound in that multi-binding.  Another
way to state this difficulty is that, unlike most algebraic data
types, names cannot be represented at the type level, so there is no
way to construct a GADT which ensures that all names in term are in a
particular list. There are three possible fixes to this problem: raise
an error when an unexpected free name is encountered; add a closedness
type, such as that used by Calcagno et al.~\cite{calcagno00}; or use a
different strategy for closed names. We choose the third option here.


Note that, although not being able to express closedness in \ourlib\
is an issue, we do not feel that it is a major problem.  In this
particular example, closedness is only needed for the lambda-bound
variables, and we still use \ourlib\ names for declaration variables;
this simplifies the definition of lambda-lifting, as it does not
require lowering code to add declaration variables to the scope
whenever a computation proceeds under a declaration binding.  Further,
there are many situations where it is not especially problematic to
choose the first option above and signal a runtime error when
encountering non-closed names. This can in fact lead to simpler code,
as the programmer need not structure the code in such a way that the
type system can track closedness.  In addition, that
authors believe both that it is possible to extend \ourlib\ to include
a notion of closedness, and also that lambda-lifting can be rewritten
to not even use an intermediate representation, thereby avoiding this
problem altogether. These possibilities are discussed below in Section
\ref{sec:conclusion}.

\begin{figure}[t]
\begin{lstlisting}[basicstyle=\ttfamily\scriptsize]
-- typed deBruijn representation of term skeletons
data STerm ctx a where
  SWeaken :: SubCtx ctx1 ctx -> STerm ctx1 a ->
             STerm ctx a
  SVar :: InCtx ctx (L a) -> STerm ctx a
  SDVar :: Name (D a) -> STerm ctx a
  SApp :: STerm ctx (a -> b) -> STerm ctx a ->
          STerm ctx b

-- converting an STerm into a DTerm (code omitted)
skelSubst :: STerm ctx a -> MapCtx Name ctx -> DTerm a

-- a STerm with a list of free variables
data FVSTerm ctx lctx a where
  FVSTerm :: FVList ctx fvs ->
             STerm (fvs :++: lctx) a ->
             FVSTerm ctx lctx a

-- separating the last bound variables (code omitted)
fvSSepLVars :: MapCtx f lctx ->
               FVSTerm (ctx :++: lctx) CtxNil a ->
               FVSTerm ctx lctx a
\end{lstlisting}
\caption{Lambda-Lifting: Term Skeletons}
\label{fig:term-skeletons}
\end{figure}


% \begin{figure}[t]
% \begin{lstlisting}[basicstyle=\ttfamily\scriptsize]
% -- typed deBruijn representation of closed terms
% data STerm ctx a where
%   SWeaken :: SubCtx ctx1 ctx -> STerm ctx1 a ->
%               STerm ctx a
%   SVar :: InCtx ctx (L a) -> STerm ctx a
%   SDVar :: Name (D a) -> STerm ctx a
%   SApp :: STerm ctx (a -> b) -> STerm ctx a ->
%            STerm ctx b

% -- substituting a STerm to get a DTerm
% dbSubst :: STerm ctx a -> MapCtx Name ctx -> DTerm a
% dbSubst (SWeaken f db) names = dbSubst db $ f names
% dbSubst (SVar inCtx) names =
%   Var $ ctxLookup inCtx names
% dbSubst (SDVar dVar) _ = DVar dVar
% dbSubst (SApp db1 db2) names =
%   App (dbSubst db1 names) (dbSubst db2 names)

% -- a STerm with a list of free variables
% data FVSTerm ctx lctx a where
%   FVSTerm :: FVList ctx fvs ->
%               STerm (fvs :++: lctx) a ->
%               FVSTerm ctx lctx a

% -- separating the last bound variables (code omitted)
% fvSSepLVars :: LCtx lctx ->
%                 FVSTerm (ctx :++: lctx) CtxNil a ->
%                 FVSTerm ctx lctx a
% \end{lstlisting}
% \caption{Lambda-Lifting: Closed Terms}
% \label{fig:closed-terms}
% \end{figure}


% \begin{itemize}
% \item define and explain \lstinline{STerm}: need it to define
%   closedness!
% \item define and explain \lstinline{dbSubst}: converts a
%   \lstinline{STerm} to a \lstinline{DTerm}
% \item \lstinline{FVSTerm}
% \item explain \lstinline{fvSSepLVars} (do not give the code!)
% \end{itemize}


Figure \ref{fig:term-skeletons} defines the type
\lstinline{STerm ctx a} of ``skeletons'' of the bodies of
lambda-abstractions. These represent terms of type \lstinline{a} with
no lambda-abstractions, whose free (lambda-bound) variables are all
listed in \lstinline{ctx}. We use a typed deBruijn representation for
lambda-bound variables to achieve this, representing each name with an
\lstinline{InCtx} proof. We also include the constructor
\lstinline{SWeaken}, which takes a term in a sub-context of
\lstinline{ctx} and weakens it to a term in context \lstinline{ctx}.
Otherwise this type mirrors \lstinline{DTerm}, without
lambda-abstractions. Note that we still use \ourlib\ names for
declaration variables, yielding a simpler type than if typed deBruijn
indices were used for these names as well.


Term skeletons can be converted back to the \lstinline{DTerm}
representation simply by substituting names for the free
(lambda-bound) variables. This is done with the function
\lstinline{skelSubst} (code omitted), which is similar to the function
\lstinline{msubst} of Section \ref{subsec:small-examples}.

The type \lstinline{FVSTerm ctx lctx a} intuitively represents a list
of free variables along with a term skeleton of type \lstinline{a}
using only the variables in this list. The free variables are given in
two pieces: the first argument to the \lstinline{FVSTerm} constructor
is an \lstinline{FVList} giving names in multi-bindings for
\lstinline{ctx}; but the skeleton can also contain names whose types
are listed in \lstinline{lctx}. Intuitively, this is because the
skeleton is the result of lambda-lifting a sequence of nested
lambda-abstractions that introduced variables for the types in
\lstinline{lctx}. Thus, when the skeleton is converted back to a
\lstinline{DTerm}, lambda-abstractions for these variables will be
added unconditionally. The remaining variables will then be abstracted
as well.

Finally, Figure \ref{fig:term-skeletons} also declares the function
\lstinline{fvSSepLVars} (code omitted). This function intuitively
takes a skeleton and its free variable list and separates out the last
variables bound in the current context. This is represented at the
type level as an operation on \lstinline{FVSTerm} which moves the
suffix of the \lstinline{ctx} type argument to be the \lstinline{lctx}
type argument. This function is called just before the body
of a lambda-abstraction is turned back into a lambda-abstraction,
in order to abstract the free variables first and then form the
lambda-abstractions found in the original term.



%%%%%%%%%%%%%%%%%%%%%%%
%%%%% Sub-Section %%%%%
%%%%%%%%%%%%%%%%%%%%%%%
\subsection{Lambda-Lifting in the Continuation Monad}
\label{subsec:lambda-lifting}

We now come to the final result, which uses all the above to perform
lambda-lifting. The key difficulty is in formulating the case for
lambdas, which intuitively requires adding top-level declarations
around the term currently being transformed. Stated differently,
lambda-lifting proceeds by structural recursion on the body of a
lambda-abstraction, but when it comes to another lambda, it must stop
the current computation, add a top-level declaration for that lambda,
and then resume the current computation inside the binding for the new
top-level declaration. To do this, we use the continuation monad
\lstinline{Cont} along with Felleisen's $\mathcal{C}$ operator
\cite{felleisen87}.  The $\mathcal{C}$ operator allows us to capture
(and abort) the current continuation, which represents a skeleton term
being computed for the current point in the code. We can then add a
top-level declaration, for the lambda being lifted, and resume the
current continuation inside that top-level declaration.  This requires
fixing the result type of the \lstinline{Cont} operator to be
\lstinline{Decls b} for some \lstinline{b}, so we can apply the
\lstinline{DeclsCons} constructor outside the current continuation.


\begin{figure}[t]
\vspace{-5pt}
\begin{lstlisting}[basicstyle=\ttfamily\scriptsize]
-- llBody return type
type LLBodyRet b ctx a =
  Cont (Decls b) (FVSTerm ctx CtxNil a)

-- proofs that a context contains only L types
data IsLType a where IsLType :: IsLType (L a)
type LCtx ctx = MapCtx IsLType ctx

-- lambda-lift the body of a function
llBody :: LCtx ctx -> Mb ctx (Term a) ->
          LLBodyRet b ctx a
llBody ctx [nuQQ| Var v |] =
  return $ FVSTerm (EmptyMC :> MbLName v)
                   (SVar InCtxBase)

llBody ctx [nuQQ| App t1 t2 |] = do
  FVSTerm fvs1 db1 <- llBody ctx t1
  FVSTerm fvs2 db2 <- llBody ctx t2
  FVUnionRet names sub1 sub2
      <- return $ fvUnion fvs1 fvs2
  return $ FVSTerm names $ SApp (SWeaken sub1 db1)
                                  (SWeaken sub2 db2)

llBody ctx [nuQQ| Lam b |] = do
  PeelRet lctx body <- return $ peelLambdas b
  llret <- llBody (ctxAppend ctx lctx) body
  FVSTerm fvs db <- return $ fvSSepLVars lctx llret
  cont $ \k ->
    DeclsCons (freeParams (fvsToLCtx fvs) $ \vs1 ->
               boundParams lctx $ \vs2 ->
               skelSubst db (ctxAppend vs1 vs2))
      $ nu $ \d ->
               k (FVSTerm fvs
                    (skelAppMultiNames (SDVar d) fvs))

-- the top-level lambda-lifting function
lambdaLift :: DTerm a -> Decls a
lambdaLift t =
  runCont (llBody EmptyMC (emptyMb t))
          (\(FVSTerm fvs db) ->
            let vs = ctxMap (\(MbLName mbn) ->
                              elimEmptyMb mbn) fvs
            in DeclsBase (skelSubst db vs))
\end{lstlisting}
\caption{Lambda-Lifting: The Main Function}
\label{fig:lambda-lifting}
\end{figure} % $


The code for lambda-lifting is given in Figure
\ref{fig:lambda-lifting}. It first defines the return type
\lstinline{LLBodyRet b ctx a} as a continuation monad with result type
\lstinline{Decls a} and whose return type is an unmapped term of type
\lstinline{a} in \lstinline{ctx}. Next is an auxiliary definition of
the type \lstinline{LCtx ctx} of type contexts containing only types
of the form \lstinline{L a}, called \lstinline{L}-contexts.
The figure then defines the main function,
\lstinline{llBody}, for lambda-lifting the bodies of lambdas.  This
function takes an auxiliary first argument, where it collects the
types of the lambda-bound variables it sees as it enters inside the
bodies of lambda-abstractions. The case for a normal variable returns
that variable as a skeleton, whose free variable list contains only
that variable.


The case for applications is slightly more complex. It first computes
the result for the function and the argument, and then unions the two
resulting free variable lists. The call to \lstinline{fvUnion} returns
the combined free variable list, along with \lstinline{SubCtx} proofs
that can be used to weaken the skeletons to use this same free
variable list.


The main complexity is in the case for lambdas, which proceeds as
follows. First it calls the auxiliary function \lstinline{peelLambdas}
(type and code are omitted), which intuitively removes all the leading
lambda-abstractions from its argument, returning a non-empty
\lstinline{L}-context for the types of the variables whose lambdas
were removed, as well as the revealed body of the lambda-abstractions.
Next, \lstinline{llBody} lambda-lifts the body, yielding a free
variable list and a skeleton term (inside the monad).  After this,
\lstinline{fvSSepLVars} is called to separate the free variables into
free variables bound by the lambda-abstractions that were removed and
those that are free in these lambda-abstractions.  Finally,
Felleisen's $\mathcal{C}$ operator, which is called \lstinline{cont}
in the \lstinline{Cont} monad, is invoked to move the current
computation inside a new binding for a declaration variable.  This new
binding is created using the \lstinline{DeclsCons} constructor.  The
first argument to \lstinline{DeclsCons}, which is the declaration
being bound, is created by calling the functions
\lstinline{freeParams} and \lstinline{boundParams} (type and code are
omitted) to build a declaration with parameter bindings corresponding
first to the free variables and then to the bound variables around the
lambda-lifted body, and then by calling \lstinline{skelSubst} to
convert the body back to a \lstinline{DTerm} using these
lambda-abstracted variables.  The second argument of
\lstinline{DeclsCons} is then created by resuming the current
continuation inside a call to \lstinline{nu}, using the newly bound
declaration variable \lstinline{d} applied to the free variables used
in the declaration as the result passed to this continuation.


Finally, Figure \ref{fig:lambda-lifting} defines
\lstinline{lambdaLift}. This function calls \lstinline{llBody} on the
input term (using \lstinline{emptyMb} to coerce it to a multi-binding
with zero names) to create a monadic computation. It then runs the
resulting computation, passing a top-level continuation which does two
things: it takes the free variable list returned by the computation
and removes all the multi-bindings for the empty context using
\lstinline{elimEmptyMb}; and it then uses this variable list to pass
to \lstinline{skelSubst}, in order to convert the returned skeleton
back to a \lstinline{DTerm}.



% %%%%%%%%%%%%%%%%%%%%%%%
% %%%%% Sub-Section %%%%%
% %%%%%%%%%%%%%%%%%%%%%%%
% \subsection{Helper Functions and Definitions}
% \label{subsec:helpers}

% We briefly introduce a number of helper datatypes and functions used
% in the code discussed below. These are given in Figure
% \ref{fig:helpers}. The first declaration in the figure gives the type
% \lstinline{Ctx ctx}, which reifies type contexts at the
% term level to allow functions to manipulate them.
% The next declaration declares the function \lstinline{mkBoundName}
% which uses these reified type contexts to create a
% multi-binding whose body contains the last name in the
% multi-binding. The code is omitted for reasons of brevity,
% but it essentially iterates over the \lstinline{Ctx} argument,
% calling \lstinline{nu} each time to bind names and then
% calling \lstinline{combineMb} to coalesce the multi-bindings.


% \begin{figure}
% \begin{lstlisting}[basicstyle=\ttfamily\scriptsize]
% -- a reification of a context at the term level
% type Ctx ctx = MapCtx Tag ctx

% -- return the last name in a context (code omitted)
% mkBoundName :: Ctx ctx ->
%                Mb (CtxCons ctx a) (Name a)

% -- proof that a type starts with L
% data IsLType a where IsLType :: IsLType (L a)

% -- proof that ctx contains only L types
% type LCtx ctx = MapCtx IsLType ctx

% -- appending an LCtx to a Ctx (code omitted)
% ctxAppendLCtx :: Ctx ctx -> LCtx lctx ->
%                  Ctx (ctx :++: lctx)

% -- L name in a multi-binding
% data MbLName ctx a where
%   MbLName :: (Mb ctx (Name (L a))) ->
%              MbLName ctx (L a)

% -- free names with their types
% data FVCtx ctx where
%   FVCtx :: MapCtx (MbLName ctx) fvs -> FVCtx ctx

% -- get the free vars of a term (code omitted)
% freeVars :: Mb ctx (DTerm a) -> FVCtx ctx

% -- type function for adding arrows to a type
% type family AddArrows ctx b
% type instance AddArrows CtxNil b = b
% type instance AddArrows (CtxCons ctx (L a)) b =
%   AddArrows ctx (a -> b)

% -- GADT for the return value of peelLambdas
% data PeelRet ctx a where
%   PeelRet :: LCtx lctx ->
%              Mb (ctx :++: lctx) (DTerm a) ->
%              PeelRet ctx (AddArrows lctx a)

% -- "peeling" the lambdas off of a term
% peelLambdas :: LCtx lctx ->
%                Mb (ctx :++: lctx) (DTerm a) ->
%                PeelRet ctx (AddArrows lctx a)
% peelLambdas lctx [nuQQ| Lam b |] =
%     peelLambdas (lctx :> IsLType)
%                 (combineMb b)
% peelLambdas lctx [nuQQ| b |] =
%     PeelRet lctx b
% \end{lstlisting}
% \caption{Helper Functions and Definitions}
% \label{fig:helpers}
% \end{figure}


% The next declarations deal with the type \lstinline{LCtx ctx} of
% reified type contexts whose elements are guaranteed to be
% \lstinline{L}-types, meaning they have the form \lstinline{L b} for
% some \lstinline{b}.  The first of these defines the type
% \lstinline{IsLType a} of proofs that the type \lstinline{a} is an
% \lstinline{L}-type. The second gives the type \lstinline{LCtx ctx} of
% proofs that the type context \lstinline{ctx} is composed entirely of
% \lstinline{L}-types. The third declaration gives the type signature of
% the function \lstinline{ctxAppendLCtx} (code omitted for brevity),
% which appends an \lstinline{LCtx lctx} to the end of a reified context
% object of type \lstinline{Ctx ctx}. This yields a reified context
% object of type \lstinline{Ctx (ctx :++: lctx)}.


% The next three declaraions in Figure \ref{fig:helpers} deal with the
% free variables of \lstinline{DTerm}s.  The type
% \lstinline{MbLName ctx a} represents a name of type \lstinline{a}
% inside a multi-binding for \lstinline{ctx}, where \lstinline{a} is
% constrained to be an \lstinline{L}-type. This type is useful in
% defining the next type, \lstinline{FVCtx ctx}, as a \lstinline{MapCtx}
% of \lstinline{MbLName ctx} over an existentially quantified type
% context \lstinline{fvs}. This latter type intuitively represents a
% free variable context as a list of names inside multi-bindings for
% \lstinline{ctx}, where the types of these names are given by
% \lstinline{fvs}.  The function \lstinline{freeVars} then maps a
% \lstinline{DTerm} in a multi-binding for \lstinline{ctx} to a free
% variable context of type \lstinline{FVCtx ctx}. We omit the definition
% of \lstinline{freeVars} for space reasons, as it is relatively
% straightforward using some of the techniques already developed in
% Section \ref{subsec:small-examples}.

% The remainder of Figure \ref{fig:helpers} defines the function
% \lstinline{peelLambdas}. This function intuitively removes all the
% \lstinline{Lam} constructors inside a multi-binding.  The result is a
% multi-binding with possibly more bound names for the lambdas that were
% removed. To define the type of this function, we first define the
% type-level function \lstinline{AddArrows ctx a} as a Haskell type
% family. This type function intuitvely takes every type
% \lstinline{L bi} in \lstinline{ctx} and adds \lstinline{bi ->} to the
% front of \lstinline{a}. The next declaration in Figure
% \ref{fig:helpers} gives the return type \lstinline{PeelRet ctx a} of
% \lstinline{peelLambdas}. This type constrains \lstinline{a} to be of
% the form \lstinline{AddArrows lctx a} for some type context
% \lstinline{lctx}. It then contains an \lstinline{LCtx} proof that all
% types in \lstinline{lctx} are \lstinline{L}-types, along with a
% \lstinline{DTerm a} --- that is, a \lstinline{DTerm} whose type has
% all the arrows given by \lstinline{lctx} peeled away --- in a
% multi-binding for both \lstinline{ctx} and \lstinline{lctx}.  Finally,
% \lstinline{peelLambdas} takes an \lstinline{LCtx} proof,
% \lstinline{lctx}, for the types alread peeled along with a term in a
% multi-binding and tests if that term is a lambda. If so, it recurses,
% adding the bound type to \lstinline{lctx} and using
% \lstinline{combineMb} to coalesce the outer bindings and the inner
% binding for the body of the lambda. Otherwise \lstinline{peelLambdas}
% returns its two arguments as a \lstinline{PeelRet}.



% %%%%%%%%%%%%%%%%%%%%%%%
% %%%%% Sub-Section %%%%%
% %%%%%%%%%%%%%%%%%%%%%%%
% \subsection{Binding Monads}
% \label{subsec:binding-monads}

% In this section we introduce a novel concept called a \emph{binding
%   monad}.  It is essentially an adaptation of the concept of a
% multi-binding to the setting of monads. Note that we use this concept
% entirely as a device for structuring the lambda-lifting code; although
% we do believe this is a deep semantic concept, complete with
% extensions of the monad laws for name-bindings, we leave this
% possibility as future work for this document.

% The interface of a binding monad, denoted by the
% % comment to prevent meta-q from wrapping the \lstinline
% \lstinline{MonadBind m} typeclass, is given in Figure
% \ref{fig:binding-monads}.  This typeclass indicates that \lstinline{m}
% is a type-level function of two arguments, a type context
% \lstinline{ctx} and a type \lstinline{a}. Intuitively, a value of type
% \lstinline{m ctx a} is a computation of type \lstinline{a} relative to
% a binding context \lstinline{ctx}.  Although the intent is that
% \lstinline{m ctx} should be a monad for any \lstinline{ctx}, it is not
% currently possible to write this requirement in GHC. The two central
% operations of a binding monad are \lstinline{nuM} and
% \lstinline{strengthenM}. The former is similar to \lstinline{nu}, in
% that it generates a fresh name and passes the name to its argument,
% \lstinline{f}. The key difference is that \lstinline{nuM} commutes the
% monad \lstinline{m ctx} outside of the binding, yielding a binding
% inside the monad.  The latter operation, \lstinline{strengthenM},
% takes a monadic computation in a given binding context and
% \emph{strengthens} it, returning a monadic computation with the last
% type in the binding context removed. The \lstinline{MonadBind}
% typeclass also contains the \lstinline{nuM'} operation, which combines
% these two operations, binding a fresh name and weakening the type
% context to contain the type of this name in the body of the binding.


% \begin{figure}
% \begin{lstlisting}[basicstyle=\ttfamily\scriptsize]
% -- binding monad type class
% class MonadBind m where
%   nuM :: (Name a -> m ctx b) ->
%          m ctx (Binding a b)
%   strengthenM :: m (CtxCons ctx a) b -> m ctx b
%   nuM' :: (Name a -> m (CtxCons ctx a) b) ->
%           m ctx (Binding a b)
%   nuM' f = nuM (strengthenM . f)

% -- a trivial instance: identity on the 2nd type
% newtype Proj2 ctx a = Proj2 { runProj2 :: a }
% instance MonadBind Proj2 where
%     nuM f = Proj2 $ nu $ runProj2 . f
%     strengthenM (Proj2 m) = Proj2 m
% \end{lstlisting}
% \caption{Binding Monads}
% \label{fig:binding-monads}
% \end{figure}


% Figure \ref{fig:binding-monads} also gives a trivial binding monad,
% \lstinline{Proj2}, which projects to its second type argument
% \lstinline{a}. For any type context \lstinline{ctx}, we have that
% \lstinline{Proj2 ctx} is just the identity monad. The \lstinline{nuM}
% operation just calls \lstinline{nu}, while \lstinline{strengthenM} is
% the identity.


% A more interesting binding monad, \lstinline{BindM}, is given in
% Figure \ref{fig:bindm}. This monad uses \emph{name mappings},
% represented with the \lstinline{NameMap ctx} type, which intuitively
% maps a single name in a multi-binding for \lstinline{ctx} to a name of
% the same type not in a multi-binding.  A list of name mappings can
% then be used to map any number of names inside multi-bindings to names
% outside of multi-bindings, using the \lstinline{mapName} function.
% (The code for \lstinline{mapName} is omitted for brevity, but it is
% essentially an association list lookup.) Name mappings can be lowered
% into an extended context using \lstinline{nameMapLower}, which
% essentially calls the \lstinline{mbLower} operation that was defined
% as a demonstration of \lstinline{mbToplevel} in
% Section~\ref{sec:matching-on-bindings}. Finally, any list of names
% inside empty multi-bindings can be mapped to themselves (outside the
% empty multi-bindings) using \lstinline{nilCtxMapApp} (code omitted),
% which essentially folds over the list of names and uses
% \lstinline{elimEmptyMb} to remove the names from their empty
% multi-bindings.


% \begin{figure}
% \begin{lstlisting}[basicstyle=\ttfamily\scriptsize]
% -- a mapping from a name-in-context to a name
% data NameMap ctx where
%   NameMap :: (Mb ctx (Name a)) ->
%              Name a -> NameMap ctx

% -- map a name (code omitted)
% mapName :: [NameMap ctx] -> Mb ctx (Name a) ->
%            Maybe (Name a)

% -- lower a mapping into another binding
% nameMapLower :: NameMap ctx ->
%                  NameMap (CtxCons ctx b)
% nameMapLower (NameMap mbN n) =
%   NameMap (mbLower mbN) n

% -- append mappings for names in CtxNil (code omitted)
% nilCtxMapApp :: FVCtx CtxNil -> [NameMap CtxNil] ->
%                 [NameMap CtxNil]

% -- the BindM monad (essentially a Reader)
% newtype BindM ctx a =
%   BindM { runBindM :: [NameMap ctx] -> a }

% -- standard Reader instances (not shown)
% instance Monad (BindM ctx)
% instance MonadReader [NameMap ctx] (BindM ctx)

% -- instance declaration for BindM
% instance MonadBind BindM where
%   nuM f = BindM $ \nmap ->
%     nu $ \n -> runBindM (f n) nmap
%   strengthenM m = BindM $ \nmap ->
%     runBindM m (map nameMapLower nmap)

% -- perform a computation with an extended name map
% withExtMap :: Mb ctx (Name a) -> Name a ->
%               BindM ctx b -> BindM ctx b
% withExtMap mbN n m =
%   local (\nmap -> NameMap mbN n : nmap) m
% \end{lstlisting} % $
% \caption{The \lstinline{BindM} Binding Monad}
% \label{fig:bindm}
% \end{figure}


% % \begin{figure}
% % \begin{lstlisting}[basicstyle=\ttfamily\scriptsize]
% % -- a mapping from a name-in-context to a name
% % data NameMap ctx where
% %   NameMap :: (Mb ctx (Name a)) ->
% %              Name a -> NameMap ctx

% % -- map a name according to a list of mappings
% % mapName :: [NameMap ctx] ->
% %            Mb ctx (Name a) -> Maybe (Name a)
% % mapName [] n = Nothing
% % mapName (NameMap bound free : rest) n =
% %     case mapName1 (mbCmpName bound n) free of
% %       Nothing -> mapName rest n
% %       Just ret -> Just ret

% % -- lower a mapping into another binding
% % nameMapLower :: NameMap ctx ->
% %                  NameMap (CtxCons ctx b)
% % nameMapLower (NameMap mbN n) =
% %   NameMap (mbLower mbN) n
% % \end{lstlisting}
% % \caption{Name Mappings}
% % \label{fig:name-mappings}
% % \end{figure}


% Using name mappings, we can then define \lstinline{BindM ctx} as the
% standard \lstinline{Reader} monad, using type
% \lstinline{[NameCtx ctx]} as the input type. (We do not directly use
% the \lstinline{Reader} definition to reduce the number of constructors
% used.)  We use the standard implementations of the monad operations,
% given in the \lstinline{Monad} typeclass, as well as the
% \lstinline{ask} and \lstinline{local} operations of the
% \lstinline{MonadReader} typeclass. (The code is omitted.)  To
% implement \lstinline{nuM} applied to \lstinline{f}, we create a
% function over name mapping list \lstinline{nmap}, to form the element
% of the \lstinline{BindM} monad, and then use the \lstinline{nu}
% operator to pass a fresh name to \lstinline{f} as the body of this
% function. The result of the call to \lstinline{f} is then in the
% \lstinline{BindM ctx} monad, and so is applied to the \lstinline{nmap}
% input argument. The \lstinline{strengthenM} operation is then defined
% by applying \lstinline{nameMapLower} to the name mappings in the input
% list, which removes any mapping for the last name bound in the current
% binding context. Note that neither of these operations can be defined
% using the standard \lstinline{Reader} operations,
% \lstinline{strengthenM} because it changes the \lstinline{ctx} type
% argument to \lstinline{BindM} (and thus returns into a different
% monad), and \lstinline{nuM} because there is no way to commute the
% binding inside the \lstinline{BindM ctx} monad without running the
% monad.


% Figure \ref{fig:bindm} also defines a useful helper function,
% \lstinline{withExtMap}. This function takes a name \lstinline{mbN} in
% a multi-binding, another name \lstinline{n} of the same type outside
% of the multi-binding, and a monadic computation \lstinline{m}. It then
% performs the computation \lstinline{m} using the extension of the
% current name mapping list to map \lstinline{mbN} to \lstinline{n}.
% This is done using the \lstinline{local} monadic operator of
% \lstinline{BindM}.


% %%%%%%%%%%%%%%%%%%%%%%%
% %%%%% Sub-Section %%%%%
% %%%%%%%%%%%%%%%%%%%%%%%
% \subsection{Unmapped Terms}
% \label{subsec:unmapped-terms}


% In this section we introduce the concept of an \emph{unmapped term}.
% This is a term that is waiting to have its bound variables mapped to
% fresh variables, thereby effectively converting it from a
% term-in-multi-binding to a term not in a multi-binding.  Unmapped
% terms are represented using the type \lstinline{MTerm ctx a}, which is
% defined in Figure \ref{fig:unmapped-terms} as a \lstinline{DTerm a}
% computation inside \lstinline{BindM ctx}. An unmapped term can be
% mapped using the \lstinline{mapMTerm} fuction. This function runs
% \lstinline{BindM} monad and passes it the empty name mapping list,
% which does not map any bound names.


% \begin{figure}[t]
% \begin{lstlisting}[basicstyle=\ttfamily\scriptsize]
% -- the MTerm type
% type MTerm ctx a = BindM ctx (DTerm a)

% -- perform the mapping
% mapMTerm :: MTerm ctx a -> DTerm a
% mapMTerm mt = runBindM mt []

% -- a Var that needs mapping
% mVar :: Mb ctx (Name (L a)) -> MTerm ctx a
% mVar v =
%   do nmap <- ask
%      return $ case mapName nmap v of
%                 Just n -> Var n
%                 Nothing -> error $ "unexpected var!"

% -- a DVar that needs mapping
% mDVar :: Mb ctx (Name (D a)) -> MTerm ctx a
% mDVar d =
%   case mbNameBoundP d of
%     Left _ -> error "unexpected definition var!"
%     Right n -> return $ DVar n

% -- an App that needs mapping
% mApp :: MTerm ctx (a -> b) -> MTerm ctx a ->
%         MTerm ctx b
% mApp mt1 mt2 =
%   mt1 >>= \t1 -> mt2 >>= \t2 -> return (App t1 t2)

% -- apply mApp to zero or more names (code omitted)
% mApplyToNames :: MTerm ctx (AddArrows lctx a) ->
%                  MapCtx (MbLName ctx) lctx ->
%                  MTerm ctx a

% -- lambda abstract over a bound name
% mLam :: Mb ctx (Name (L a)) -> MTerm ctx b ->
%         MTerm ctx (a -> b)
% mLam mbN mt =
%   nuM (\n -> withExtMap mbN n mt) >>= (return . Lam)

% -- call mLam multiple times (code omitted)
% mLamMulti :: MapCtx (MbLName ctx) fvs ->
%              MTerm ctx a ->
%              MTerm ctx (AddArrows fvs a)

% -- lambda abstract and remove the tail of ctx
% mLam' :: Ctx ctx -> MTerm (CtxCons ctx (L b)) a ->
%          MTerm ctx (b -> a)
% mLam' ctx mt =
%   nuM' (\n -> withExtMap (mkBoundName ctx) n mt) >>=
%   (return . Lam)

% -- call mLam' multiple times (code omitted)
% mLamMulti' :: Ctx ctx -> LCtx lctx ->
%               MTerm (ctx :++: lctx) a ->
%               MTerm ctx (AddArrows lctx a)

% -- map the given free vars in an MTerm to themselves
% mTermWithFVs :: FVCtx CtxNil -> MTerm CtxNil a ->
%                 MTerm CtxNil a
% mTermWithFVs fvs = local (nilCtxMapApp fvs)
% \end{lstlisting} % $
% \caption{Unmapped Terms}
% \label{fig:unmapped-terms}
% \end{figure}


% Figure \ref{fig:unmapped-terms} then gives a number of functions to
% build unmapped terms.  The \lstinline{mVar} function takes a
% lambda-bound name in a multi-binding and intuitively builds the term
% resulting from mapping that name and then applying the \lstinline{Var}
% constructor.  To perform the mapping, \lstinline{mVar} uses the
% \lstinline{ask} monad operator to retrieve the current name mapping
% list, and then it calls \lstinline{mapName} to map the
% name-in-multi-binding.  If the mapping is successful, the result is
% \lstinline{Var} applied to the result; otherwise an error is
% signaled. The \lstinline{mDVar} is similar, except it does not
% expect declaration-bound variables to be bound, so it uses
% \lstinline{mbBoundNameP} to attempt to lift the name out of the
% multi-binding, applying \lstinline{DVar} if successful and signaling
% an error otherwise. \lstinline{mApp} applies an unmapped term to
% another by using the \lstinline{>>=} operator to get the mapped
% results; \lstinline{mApplyToNames} (code not shown) calls
% \lstinline{mApp} zero or more times to apply a function to a list
% of (unmapped) variables.


% To handle lambda-abstractions requires the \lstinline{nuM} operator of
% our binding monad. The \lstinline{mLam} function takes a name
% \lstinline{mbN} of type \lstinline{L a} inside a multi-binding, along
% with an unmapped term of type \lstinline{b}, and forms an unmapped
% term \lstinline{mt} of type \lstinline{a -> b}. It does this by first
% calling \lstinline{nuM} to bind a fresh name \lstinline{n} of type
% \lstinline{L a}. Inside this binding, it applies
% \lstinline{withExtMap} in order to perform the computation
% \lstinline{mt} using an extension of the current name mapping list to
% map \lstinline{mbN} to \lstinline{n}. The \lstinline{mLam} function
% then uses the monadic bind \lstinline{>>=} to apply the
% \lstinline{Lam} constructor to the result. The \lstinline{mLamMulti}
% function (code omitted for brevity) calls \lstinline{mLam} multiple
% times, using a list of names of types \lstinline{fvs} in
% multi-bindings and yielding an unmapped term of type
% \lstinline{AddArrows fvs} applied to the input type.

% The \lstinline{mLam'} function is similar except that it uses
% \lstinline{nuM'}. Thus it takes an unmapped term in the extended
% context \lstinline{CtxCons ctx b} and returns an unmapped term in the
% context \lstinline{ctx}.  This function also binds a fresh name of
% type \lstinline{b} to which to map the last name, of type
% \lstinline{b}, in the extended context. To construct the extended
% mapping, \lstinline{mLam'} calls the \lstinline{mkBoundName} function
% declared in Figure \ref{fig:helpers}.  This requires an argument of
% type \lstinline{Ctx ctx}, which in fact causes the main lambda-lifting
% functions given below in Section \ref{subsec:lambda-lifting} to all
% require a \lstinline{Ctx} argument. The \lstinline{mLamMulti'}
% function calls \lstinline{mLam'} multiple times, requiring only an
% \lstinline{LCtx} argument to specify the extension to the current
% context.

% The final operation of Figure \ref{fig:unmapped-terms},
% \lstinline{mTermWithFVs}, takes a list of free variables
% \lstinline{fvs} and an \lstinline{MTerm} \lstinline{mt} and
% intuitively creates an unmapped term that passes identity mappings for
% \lstinline{fvs} to \lstinline{mt}. To achieve this, the
% \lstinline{local} monadic operation is used to append mappings for
% \lstinline{fvs} to the current name mapping list.  The append
% operation is performed by the \lstinline{nilCtxMapApp} function
% declared in Figure \ref{fig:bindm}.



%%%%%%%%%%%%%%%%%%%%%%%%%%%%%%%%%%%%%%%%
%%%%%%%%%%     Section        %%%%%%%%%%
%%%%%%%%%%%%%%%%%%%%%%%%%%%%%%%%%%%%%%%%
\section{Inside \ourlib}
\label{sec:implementation}

We now describe the implementation of \ourlib. The key declarations
are given in Figure \ref{fig:implementation}; the remaining
operations are straightforward, and so are omitted for brevity.
Internally, the type \lstinline{Name a} is represented as an integer,
while the type \lstinline{Mb ctx b} is a pair of a list of integers,
corresponding to the names bound by the multi-binding, along with the
body \lstinline{b}. To generate fresh names, we define a function
\lstinline{fresh_name} which increments a global pointer,
\lstinline{counter}, and returns the current value. The \lstinline{nu}
operation then calls \lstinline{fresh_name} to create a fresh name and
then passes it to its argument \lstinline{f}. It then uses Haskell's
\lstinline{seq} operator to ensure that the fresh name is evaluated
strictly.


\begin{figure}[t]
\begin{lstlisting}[basicstyle=\ttfamily\scriptsize]
-- name and multi-binding types
newtype Name a = MkName Int
data Mb ctx b  = MkMb [Int] b

-- the counter for fresh names
counter :: IORef Int
{-# NOINLINE counter #-}
counter = unsafePerformIO (newIORef 0)

-- return a fresh integer
fresh_name :: a -> Int
fresh_name a = unsafePerformIO $ do 
  dummyRef <- newIORef a
  x <- readIORef counter
  writeIORef counter (x+1)
  return x

-- implementation of nu
nu f = let n = fresh_name f in
       n `seq` MkMb [n] (f (MkName n))

-- implementation of cmpName
cmpName (MkName n1) (MkName n2) =
  if n1 == n2 then
    Just $ unsafeCoerce Refl
  else
    Nothing

-- create an InCtx of length n (code omitted)
unsafeLookupCtx :: Int -> InCtx ctx a

-- implementation of mbNameBoundP
mbNameBoundP (MkMb names (MkName n)) =
  case (elemIndex n names) of
    Nothing -> Right (MkName n)
    Just i -> Left (unsafeLookupCtx i)

-- super combinator type
newtype SuperComb a = SuperComb { unSuperComb :: a }

-- test if all names are bound (code omitted)
isSuperComb :: TH.Exp -> Bool

-- safely introduce the SuperComb constructor
superComb :: TH.ExpQ -> TH.ExpQ
superComb m = do
  e <- m
  let msg = "not a super combinator:\n\t"
  if isSuperComb e
    then return (TH.AppE (TH.ConE 'SuperComb) e)
    else fail (msg ++ TH.pprint e)

-- implementation of mbToplevel
mbToplevel f (MkMb names i) =
  MkMb names (unSuperComb f i)
\end{lstlisting}
\caption{Implementation of \ourlib}
\label{fig:implementation}
\end{figure}

There is a little bit of trickery required to get \lstinline{nu} to
work correctly in the face of compiler optimizations, because of the
uses of \lstinline{unsafePerformIO}.  First, the
\lstinline{fresh_name} function takes a dummy argument \lstinline{a}
and allocates a pointer to it. This is to make sure the monadic
computation looks like it depends on \lstinline{a}, so that the
compiler does not decide the monadic computation is a constant that can be
lifted out of the body of \lstinline{fresh_name}.  Also,
\lstinline{counter} is marked with the \lstinline{NOINLINE} pragma so
the compiler cannot accidentally inline the code for creating
\lstinline{counter}, causing a new pointer to be allocated for each
use. The \lstinline{fresh_name} function does not require this pragma,
however, because its monadic computation is intended to be performed
each time it is called. Note that \lstinline{nu} does allow common
subexpression elimination (CSE) because, if two calls to
\lstinline{nu} use the same argument \lstinline{f}, then it is ok for
them to use the same integer for their bound names, as we are ensuring
$\alpha$-Equivalence.


The \lstinline{cmpName} operation is relatively straightforward, using
\lstinline{==} to compare two names, represented by integers, for
equality.  If the names are equal, however, there is no guarantee from
the Haskell type system that their associated types are equal.  Since
each name always has at most a single type, however, we know that the
two types must be the same, and we thus call the Haskell cast function
\lstinline{unsafeCoerce} to cast our proof \lstinline{Refl} of type
equality to a proof of equality of the types associated with the
names. The \lstinline{mbNameBoundP} function compares a name with the
names bound in a multi-binding using \lstinline{elemIndex} to get the
position of the name in the list. If the name is not in the list we
know it can safely be removed from the multi-binding, so it is
returned, otherwise \lstinline{unsafeLookupCtx} is called (code
omitted), which contains a similar call to \lstinline{unsafeCoerce} to
cast an \lstinline{InCtx} proof to the right type.


As discussed in Section \ref{subsec:mbtoplevel}, the
\lstinline{mbToplevel} operator uses the type \lstinline{SuperComb a}
type to ensure safety, by only allowing super combinators to be
applied under multi-bindings. This type is isomorphic to
\lstinline{a}, with the \lstinline{SuperComb} constructor applied, and
\lstinline{mbToplevel} simply removes this constructor from its first
input and applies the resulting function to the body of the
$\nu$-abstraction in the second input. To ensure that the elements of
this type are always super combinators, \ourlib\ hides this data type
(as with \lstinline{Name} and \lstinline{Mb}), thus only allowing
super combinators to be created by the \lstinline{superComb}
function. This function creates a monadic computation, inside Template
Haskell's \lstinline{Q} monad, which first calls the
\lstinline{isSuperComb} function to determine if the argument is in
fact a super combinator. The code for this latter function is omitted,
but it essentially traverses the input and determines that all Haskell
variables that occur are either global or are bound inside the input
term.  (This is done by checking that all constructors for the
Template Haskell \lstinline{Name} type are either \lstinline{NameG} or
\lstinline{NameU}.) If this check passes, then the
\lstinline{superComb} function simply wraps the input expression in an
application of the (hidden) \lstinline{SuperComb} constructor;
otherwise a compile-time error is signaled.

To implement pattern-matching under bindings, the quasi-quoter
\lstinline{nuQQ} contains a slightly simplified parser for Haskell
patterns, transforming them as follows. First, the whole pattern is
wrapped in an application of the \lstinline{MkMb} constructor applied
to a fresh variable name, say \lstinline{names}. Then each variable
\lstinline{x} in the pattern is replaced by the \emph{view pattern}
\lstinline{(MkMb names -> x)}. Recall that a view pattern
\lstinline{(f -> P)} first applies \lstinline{f} to its argument and
then matches against \lstinline{P}. In our case, the argument to each
of the view patterns is the corresponding subterm of the entire
$\nu$-pattern's argument. Thus \lstinline{nuQQ} generates a pattern
that distributes the multi-binding context of its argument's
constructor into the subterms. The user cannot directly refer to any
of the subterms outside of their multi-bindings, since such values are
never even bound to a name. Furthermore, each subterm's multi-binding
is also constrained to have the same phantom type argument as the
multi-binding being matched.



%%%%%%%%%%%%%%%%%%%%%%%%%%%%%%%%%%%%%%%%
%%%%%%%%%%     Section        %%%%%%%%%%
%%%%%%%%%%%%%%%%%%%%%%%%%%%%%%%%%%%%%%%%
\section{Proving Safety}
\label{sec:safety}

In this section, we argue that the multi-bindings implemented by
\ourlib\ satisfy Freshness, $\alpha$-Equivalence, Scoping, and Typing.
These properties ensure in turn that \lstinline{nu} is effectively
pure and that the unsafe casts in the \ourlib\ implementation are
always safe. The former holds by $\alpha$-Equivalence, which ensures
that the user cannot observe the imperative actions of \lstinline{nu}
other than the fact that different bindings yield unequal names (as
required by Freshness). The latter holds by Typing, since the only
unsafe casts in the \ourlib\ implementation are used to prove that two
types for the same name are equal.

To demonstrate that these properties hold, we first introduce the
\emph{$\alpha$-well-formed} terms, an extension of Scoping that is
needed to prove $\alpha$-Equivalence. We then prove two lemmas, that
the $\alpha$-well-formed terms are closed under reduction and that
$\alpha$-equivalence of \emph{$\alpha$-well-formed} terms is a
bisimulation for reduction; we use a call-by-need notion of reduction
to model Haskell's lazy evaluation. Scoping and $\alpha$-Equivalence
follow immediately. Freshness is a consequence of these lemmas
because, although it is possible to reduce to a term that uses the
same name for nested bindings, this cannot be \emph{observed} by a
program. This is because any reduction on such a term is equivalent to
a reduction on an $\alpha$-equivalent term where the inner binding
uses a different name. The fourth property, Typing, then follows by a
straightforward proof that the typing judgment is also preserved under
$\alpha$-equivalence. For reasons of brevity, however, we do not give
a type system nor prove this fact here but it is a straightforward
proof using standard techniques. We also omit proofs of lemmas for
brevity, but these are all by straightforward induction on terms.


We begin by formalizing an operational semantics for Haskell with
\ourlib, given in Figure \ref{fig:opsem}. This is based on the
call-by-need semantics of Ariola et al.~\cite{ariola95}. The top half
of the figure defines five syntactic classes: the operators \op, the
terms $M$, the values $V$, the \letname-contexts $L$, and the
evaluation contexts $E$. The operators include (abbreviated names for)
the \lstinline{nu}, \lstinline{cmpName}, \lstinline{mbToplevel}, and
\lstinline{mbNameBoundP} operators of \ourlib, along with a $\case[]$
construct for pattern-matching. The \mbTop\ operator always
includes a top-level function $F$, which is simply a closed term
(having no free names or variables), while the $\case[]$ operator is
always annotated with a list of constructors over which it
matches. (Normally the constructors are determined by the type of the
scrutinee, but we are avoiding giving a type system here, for
brevity.) Note that, for brevity, we are actually formalizing only a
simplified version of \ourlib\ with only single bindings instead of
multi-bindings; the results generalize easily to the case with
multi-bindings. Thus we do not include the operators related to
multi-bindings.


\begin{figure}
\centering
\begin{small}
\begin{math}
\begin{array}{@{}c@{}}

%%% Syntax
\begin{array}{@{}l@{\hspace{5pt}}c@{\hspace{5pt}}l@{}}
\op & ::= & \nuOp \bor \cmp \bor \boundP \bor \mbTop\;F \bor \case
%\emp \bor \elimEmp
%\\
%& & \borstar \sep \bor \comb \bor \case
\\

M & ::= & x
\begin{array}[t]{@{}l@{}}
\bor c \bor n \bor \op \bor M\;M \bor \lamabsnot{x}{M} \bor \nuabs{n}{M}\\
\bor \lete{x}{M}{M}
\end{array}
\\

V & ::= & \lamabsnot{x}{M} \bor n \bor c\;\vec{M} \bor \nuabs{n}{M}
\\

L & ::= & [] \bor \lete{x}{M}{L}
\\

E & ::= & []
\begin{array}[t]{@{}l@{}}
\bor E\;M \bor \lete{x}{M}{E} \bor \lete{x}{E}{\tctxt{E}{x}}\\
\bor \op\;E \bor \cmp\;n\;E \bor \boundP\;(\nuabs{n}{E}) \bor \case\;(\nuabs{n}{E})
\end{array}

\end{array}

\\ \\
%%% Semantics

\begin{array}{@{}rcl@{}}

%% context rule
\tctxt{E}{M} & \rrto & \tctxt{E}{M'}\;\text{(for $M\rrto M'$)}

\\[5pt]

%% operator rules
\nuOp\;V & \rrto & \nuabs{n}{V\;n}\;\text{(for fresh $n$)}\\

\cmp\;n\;n & \rrto & \True\\

\cmp\;n\;n' & \rrto & \False\;\text{(for $n\not=n'$)}\\

\boundP\;(\nuabs{n}{\tctxt{L}{n}}) & \rrto & \Nothing\\

\boundP\;(\nuabs{n}{\tctxt{L}{n'}}) & \rrto & \Just\;n'\;\text{(for $n\not=n'$)}\\

\mbTop\;F\;(\nuabs{n}{M}) & \rrto & \nuabs{n}{F\;M}\\

\case\;(c_i\;\vec{M})\;\vec{N} & \rrto & N_i\;\vec{M}\\

\case\;(\nuabs{n}{\tctxt{L}{c_i\;\vec{M}}})\;\vec{N} & \rrto & N_i\;(\nuabs{n}{\tctxt{L}{\vec{M}}})

\\[5pt]

%% Ariola rules
(\lamabsnot{x}{M})\;N & \rrto & \lete{x}{N}{M}\\

\lete{x}{V}{\tctxt{E}{x}} & \rrto & \lete{x}{V}{\tctxt{E}{V}}\\

(\lete{x}{M}{A})\;N & \rrto & \lete{x}{M}{A\;N}\\[2pt]

\multicolumn{3}{l}{
\lete{x}{(\lete{y}{M}{A})}{\tctxt{E}{x}}
}
\\
 & \rrto & \lete{y}{M}{\lete{x}{A}{\tctxt{E}{x}}}

\end{array}

\end{array}
\end{math}
\end{small}
\caption{Operational Semantics of Haskell with \ourlib}
\label{fig:opsem}
\end{figure}


The terms include variables $x$, constructors $c$, names $n$,
operators, applications $M\;M$, $\lambda$-abstractions
$\lamabsnot{x}{M}$, $\nu$-abstractions $\nuabs{n}{M}$, and
\letname-expressions. The \letname-expressions are included to model
lazy environments, where the value for a variable $x$, bound by a
\letname, is only substituted for $x$ when the value of $x$ is needed.
This is discussed in more detail by Ariola et al.~\cite{ariola95}.
Terms are considered equal up to renaming of bound variables $x$ but
not of bound names $n$, just as in the actual implementation of
\ourlib. We write this equality as $\equiv$. We also write $\equiva$
for equality up to renaming of bound variables and names, where
``$\alpha$-equivalence'' in the below (without the capital ``E'')
refers to this latter relation.  The values include the
terms that represent valid results of computations.

The remaining two syntactic classes define two classes of \emph{term
  contexts}. These represent terms with a single hole, written $[]$. If
$C$ is a term context, we write $\tctxt{C}{M}$ for the result of
replacing (in a non-capture-avoiding manner) the hole in $C$ with $M$.
The \letname-contexts $L$ are holes in the bodies of zero or more
\letname-expressions. These essentially represent environments.  The
evaluation contexts are used to define where evaluation is allowed to
take place: at the top of a term; to the left of an application; in
the body of a \letname-expression; in the value bound by a
\letname-expression when that value is needed in the body; in the
argument of an operator, or the second argument of the two-argument
operator \cmp; or inside a $\nu$-abstraction that is the argument to
\boundP\ or \case. We also define the \emph{answers} as the set of
terms $\tctxt{L}{V}$ of values in the bodies of zero or more
\letname-expressions. We write $A$ for answers.


The bottom half of Figure \ref{fig:opsem} give reduction rules for
evaluating terms. These are split into: the first rule, which closes
reduction under evaluation contexts; the rules for operators; and the
rules for functions and environments given by Ariola et
al.~\cite{ariola95}. Most of the rules for operators are
straightforward.  \nuOp\ creates a binding with a fresh name. \cmp\
returns \True\ or \False\ (which are constructors that are assumed to
exist), where we are again simplifying \ourlib\ in a non-essential way
by omitting type equality proofs. Similarly, \boundP\ returns either
\Nothing\ or $\Just\;n$ for the name $n$ in the body of the argument.
Note that \letname-bindings are not moved outside of
$\nu$-abstractions, since they could depend on the name bound by the $\nu$, so
the name in the argument matched by \boundP\ could be in the body of
zero or \letname-expressions, which is represented by the use of the
\letname-context $L$. $\mbTop\;F$ applies $F$ to the body of a
$\nu$-abstraction, while \case\ matches on constructor applications,
applying the $(i+1)$th argument (which represents the $i$th case of
the pattern-match) to the matched subterms for the $i$th constructor.
Note that \case[] can also match constructors inside $\nu$-abstractions,
so again we include the \letname-context $L$ inside the
$\nu$-abstraction in the argument.

The last rules in the figure turn an application into a body
in an environment, substitute a bound value in an environment for a
variable when the value is needed, and commute environments upwards in
a term as needed. The interested reader can consult the work of Ariola
et al.~\cite{ariola95} for more discussion of these rules.


We implicitly assume in the below that all occurrences of operators in
terms are \emph{fully applied}, meaning that they occur, possibly
inside an environment, applied to an argument; i.e., they have the
form form $\tctxt{L}{\op}\;M$ for some $M$. The \cmp\ operator is
binary, so must have two arguments, meaning occurrences have the form
$\tctxt{L_1}{\tctxt{L_2}{\cmp}\;M_1}\;M_2$.  We also define the
\emph{normal forms} as the terms that do not reduce, and the
\emph{ill-formed terms} as those that contain an application of a name
or a $\nu$-abstraction. We can then prove the following two lemmas,
which state that our semantics behaves how we want it to and that it
does not have too few or two many cases.

\begin{lemma}[Normal Forms]
  Any closed, fully applied, normal form $M$ is either an answer or is
  an ill-formed term.
\end{lemma}

\begin{lemma}[Determinism]
  For any $M$, there is at most one $M'$ (up to $\equiva$) such that
  $M\rrto M'$.
\end{lemma}


We would like to prove that reduction is insensitive to
$\alpha$-equivalence, as discussed above. The one rule that is
problematic, however, is substitution (the second rule from Ariola et
al.), as substitution is not capture-avoiding for names. As an
example, consider the following term:
\[
\lete{x}{n}{\boundP\;(\nuabs{n}{x})}
\]
Reducing this term twice substitutes $n$ for $x$ and then reduces the
application of \boundP\ to \True. If we $\alpha$-convert the
$\nu$-abstraction, however, the term reduces to \False.
To avoid this situtation, we must ensure that no substitution can
ever cause a name to be substituted under a binder for it.
In order to prevent this, however, we must also preclude
$\nu$-abstractions from occurring inside $\lambda$-abstractions,
or we could have the term
\[
(\lamabsnot{x}{\boundP\;(\nuabs{n}{x})})\;n
\]
which reduces to the above in one step.  Thus we make the following
definition:

\begin{definition}
  $M$ is \emph{$\alpha$-well-formed}, written \WFa{M}, iff:
  \begin{enumerate}
  \item All variables $x$ and names $n$ occur bound;
  \item No $\nu$-abstractions occur inside $\lambda$-abstractions; and
  \item If we fully substitute $M$ then we do not substitute
    a name $n$ into a $\nu$-abstraction for $n$, where
    \emph{fully substituting} $M$ means repeatedly replacing occurrences
    of $x$ by $N$ in the bodies of $\letname\;x=N$ bindings.
  \end{enumerate}
\end{definition}


Using this definition, we can now prove the main lemmas of this
section:

\begin{lemma}[Preservation of \WFa{}]
  If $M\rrto M'$ for $\WFa{M}$ then $\WFa{M'}$.
\end{lemma}

\begin{lemma}[Bisimulation for \WFa{}]
  If $M\equiva N$ for $\WFa{M}$ and $\WFa{N}$, and if $M\rrto M'$,
  then $N\rrto N'$ for some $N'\equiva M'$.
\end{lemma}


%%%%%%%%%%%%%%%%%%%%%%%%%%%%%%%%%%%%%%%%
%%%%%%%%%%     Section        %%%%%%%%%%
%%%%%%%%%%%%%%%%%%%%%%%%%%%%%%%%%%%%%%%%
\section{Related Work}
\label{sec:related-work}

There are a number of approaches to representing name-bindings in the
literature. The oldest approach, deBruijn indices \cite{debruijn72},
represents variables as the number of binders between the occurrence
and the binding of the variable. A key benefit of deBruijn indices is
that they are conceptually simple. Operations like substitution that
move terms inside other bindings, however, require subtle index math
to update the number used by bound variables. This index math is often
a source of bugs, as it is difficult to get right.  Although it is
possible to extend deBruijn indices to associate type information with
bound names \cite{sheard05,atkey09}, this approach becomes an encoding
of ``term of type $A$ in context $\Gamma$'' instead of just ``term of
type $A$''. This approach requires heavy use of functions to
manipulate the context $\Gamma$, which effectively perform the same
index math as untyped deBruijn indices but on proof objects.  This
makes programs that manipulate bindings more awkward to write.

The Locally Nameless approach of McBride and McKinna \cite{mcbride04}
uses deBruijn indices only for bound names, using a different type
(such as strings) for free variables. McBride and McKinna show that
the index math required by deBruijn indices can be limited to two
operations, \lstinline{abstract} and \lstinline{instantiate}, for
Locally Nameless representations, thereby greatly easing the burden of
the programmer. The authors do not know of any research that has
associated types with names under the Locally Nameless approach, but
such an approach would either be similar to typed deBruijn indices,
requiring a context $\Gamma$ for the free variables of a term, or it
would not constrain the free variables of a term in any way.


There has been much research into representations based on
Higher-Order Abstract Syntax (HOAS). This terminology, coined by
Pfenning and Elliott \cite{pe88}, refers to the use of functions in the
host language to represent name-bindings. The benefits of HOAS are
that functional languages already ensure $\alpha$-Equivalence,
Scoping, and Typing for functions, and so these properties are ``for
free'' in HOAS.  Substitution is also for free, using function
application in the host language.  In a sense, however, HOAS does not
give Freshness, because variables in the host language cannot be
considered unequal. This leads to one of the main drawbacks of HOAS,
that variables cannot be compared. The other main drawback is that it
is impossible to pattern-match under host language functions, since
the body of a function can depend on its input.

A number of techniques have been investigated for getting around the
limitations of HOAS. Meijer and Hutton \cite{meijer-hutton95} showed
how to define operations over HOAS while simultaneously defining their
inverses.  Fegaras and Sheard \cite{fegaras96} showed that an inverse
is not needed if name-binding functions are guaranteed to be
parametric (meaning that they do not examine their arguments) and if a
free variable constructor is added to datatypes being examined.
Washburn and Weirich \cite{washburn03} built on this approach, showing
how to use parametric polymorphism to ensure that functions meant to
represent name-bindings are parametric. This is difficult to use in
practice, however, because it requires operations on HOAS encodings to
have complex, higher-order types.  Further, these approaches can only
write operations which are folds, which cannot express operations like
equality tests or getting the free variables of a term.

A number of approaches have also shown how to add support for HOAS
into the host language itself. These include $\alpha$-Prolog
\cite{miller91}, Twelf \cite{PfS98}, the modal calculus of
Sch\"{u}rmann et al.~\cite{schurmann01}, Bedwyr \cite{baelde07},
Beluga \cite{pientka08}, and Delphin \cite{poswolsky08}. These are all
special-purpose languages, however, and it is not clear how to use the
approaches in more mainstream languages like Haskell.


%FIXME: alpha-Caml


%%%%%%%%%%%%%%%%%%%%%%%%%%%%%%%%%%%%%%%%
%%%%%%%%%%     Section        %%%%%%%%%%
%%%%%%%%%%%%%%%%%%%%%%%%%%%%%%%%%%%%%%%%
\section{Conclusion}
\label{sec:conclusion}

In this paper, we have described a novel library, \ourlib, for
higher-order encodings. Higher-order encodings make it possible to
define adequate encodings of typed programming languages, thus
eliminating whole classes of bugs in compilers, interpreters, and
other language processing tools. Specifically, functions over adequate
encodings of typed programming languages are \emph{statically
  guaranteed} to preserve well-typedness of the programs being
manipulated. \ourlib\ represents name-bindings as pairs of the name
being bound and the body of the binding, thus allowing
pattern-matching on the bodies of bindings to be easily defined in the
host language. In order to ensure Freshness, $\alpha$-Equivalence,
Scoping, and Typing, \ourlib\ hides this representation, exposing a
set of combinators to allow the user to manipulate name-bindings.
We then show how \ourlib\ can be used for more complex examples,
such as lambda-lifting.


A natural question to ask at this point is: how much farther can we
push \ourlib? There are a number of things it still cannot do. As one
example, the CNIC language (which was the motivation for \ourlib)
contains an elimination form for name-bindings.  This elimination form
allows a user to remove a name-binding from around its body by
supplying a fresh name to replace the bound name. Ensuring that the
user-supplied name is fresh seems to require support in the Haskell
type system that cannot easily be described by a library.


Another useful feature of \ourlib\ would be to add support for
closedness, as discussed in Section \ref{subsec:term-skeletons}.
Since functions can be applied inside name-bindings, their inputs
could always potentially have free variables; this is captured by the
type of \lstinline{mbNameBoundP}, which always has the possibility
that a name can be removed from a multi-binding because it is not
bound there. It would be useful to have a closedness type, to indicate
statically that a value does not have free names.  This would make
functions that cannot handle free names, such as an interpreter (which
does not know what to do with a free variable), a little nicer to
write; currently, such functions must include a case for free
variables which simply reports an error.  Further, closedness types
would allow imperative operations to occur inside name-bindings, if we
require all writes to pointers to store only closed values to prevent
scope extrusion \cite{calcagno00}.  The fact that imperative
operations cannot occur inside name-bindings in the current \ourlib\
can be expressed by the fact that the \lstinline{IO} and
\lstinline{Mb} type constructors do not commute. It is well known that
modal type systems are useful for expressing closedness, but this
would also require special support from the Haskell type system. A
different approach might be to use environment classifiers
\cite{taha03}, which would not require any modifications to Haskell.


\acks This work was supported in part by NSF award CCF-0964520. Any
opinions, findings and conclusions or recommendations expressed in
this material are those of the authors and do not necessarily
reflect those of the National Science Foundation.  The authors would
like to acknowledge the PL group at KU for lots of great feedback
after reading a draft of this paper in their reading group. We would
also like to thank Stephanie Weirich for helpful discussions and Simon
Peyton-Jones for suggesting the lambda-lifting example.


\bibliographystyle{abbrvnat}
\bibliography{bib.bib}


\end{document}

% Local Variables:
% fill-column: 70
% End:
